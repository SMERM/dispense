%
% $Id: complex.tex 14 2014-02-04 22:36:30Z nicb $
%
\svnInfo $Id: complex.tex 14 2014-02-04 22:36:30Z nicb $

\section{Cosa sono i numeri complessi?\label{sec:cosa}}

\begin{itemize}

	\item qual'\`e il risultato dell'equazione $x^2 + 1 = 0$?

	\item per trovare la soluzione a questa domanda i matematici hanno esteso il
	dominio dei numeri reali con i numeri \emph{immaginari}, detti anche numeri
	\emph{complessi}

  \item propriet\`a dei numeri complessi:

		\begin{itemize}
	
	  	\item si tratta di numeri bidimensionali
	          (costituiti di due parti, una parte reale e una parte \emph{immaginaria})

			\item si possono immaginare quindi come numeri che invece di trovarsi su
						una retta si trovano su un piano

			\item pertanto si perde la possibilit\`a di \emph{ordinarli} in maniera
			      semplice ($==$ non ha senso dire che ``un numero complesso \`e
						pi\`u grande o pi\`u piccolo di un altro'')

			\item operazioni sui numeri complessi:

					\begin{description}

					  \item[coniugazione] il \emph{complesso coniugato} di un
						numero complesso \`e lo stesso numero con la parte immaginaria
						invertita:

							\begin{equation}
								z = x + i y, \bar{z} = x - i y
							\end{equation}

						la moltiplicazione di un numero complesso con il suo complesso
						coniugato d\`a luogo a un numero reale che \`e il quadrato della
						parte reale sommato al quadrato della parte immaginaria

		          \begin{equation}
							 	z \bar{z} = ( x + i y ) ( x - i y ) = x^2 + y^2 + i x y - i x y = x^2 + y^2
		          \end{equation}
						

						\item[addizione/sottrazione] si aggiungono
						e si sottraggono separatamente: parte reale con parte reale e parte immaginaria con parte immaginaria 
						\item[moltiplicazione] si moltiplicano come nella moltiplicazione
						di binomi, ricordando per\`o che $i^2 = -1$:

						\begin{equation}
						   a = x_1 + iy_1,\quad b = x_2 + iy_2\quad\\
							 a \times b = x_1 x_2 - y_1 y_2 + i \left ( x_1 y_2 x_2 y_1 \right )
		        \end{equation}

						\item[divisione] le divisioni sono definite negli termini delle
						moltiplicazioni, moltiplicando entrambi i fattori per il complesso
						coniugato del denominatore:

						\begin{equation}
						   z_1 = x_1 + i y_1,\quad z_2 = x_2 + i y_2
		 				\end{equation}
		 				\begin{equation}
							 \frac{z_1}{z_2} = \frac{x_1 + i y_1}{x_2 + i y_2}
		         \end{equation}
		         \begin{equation}
							 \frac{(x_1 + i y_1) \times (x_2 - i y_2)}{(x_2 + i y_2) \times (x_2 - i y_2)}\\
							 = \left ( \frac{x_1 x_2 + y_1 y_2 }{x_2^2 + y_2^2} \right ) + i \left ( \frac{y_1 x_2 - x_1 y_2}{x_2^2 + y_2^2} \right )
		 				\end{equation}
		 				\begin{equation}
							 \frac{(x_1 x_2 + y_1 y_2) - i (y_1 x_2 + x_1 y_2)}{x_2^2 + y_2^2}
		        \end{equation}

					\end{description}

			\end{itemize}
							
\end{itemize}

\section{Le scomposizioni in serie}

\begin{itemize}

  \item come si calcolano al computer i valori di
    $e^x$, $sin(x)$, $cos(x)$, ecc.?

	\item si calcolano in forma approssimata \emph{scomponendoli in serie}:

		\begin{equation}
    	  e^x = 1 + x + \frac{x^2}{2!} + \frac{x^3}{3!} + \frac{x^4}{4!} + \ldots + \frac{x^n}{n!}
		\end{equation}
		\begin{equation}
       sin(x) = x - \frac{x^3}{3!} + \frac{x^5}{5!} - \frac{x^7}{7!} + \ldots
		\end{equation}
		\begin{equation}
       cos(x) = 1 - \frac{x^2}{2!} + \frac{x^4}{4!} - \frac{x^6}{6!} + \ldots
		\end{equation}

    se si usano i numeri complessi, si ottiene la formula di Eulero:

		\begin{equation}
    	e^{ix} = cos(x) + i sin(x)
		\end{equation}

		(ossia sommando le scomposizioni in serie di $cos(x)$ e $i sin(x)$ e
		si ottiene la scomposizione in serie di $e^x$)

		dato che $i = 0 + i$, $i^2 = -1$, $i^3 = 0 - i$, $i^4 = 1$, $i^5 = i$, $i^6 = -1$, ecc.

    quindi: $e^x$ va all'infinito, ma $e^{ix}$ oscilla tra $+1$ e $-1$

    dato che anche $i^n$ oscilla invece di andare all'infinito

\end{itemize}

\section{Altre propriet\`a dei numeri complessi}

\begin{itemize}

  \item parte reale:

		\begin{equation}
			z = x + i y\nonumber
		 \end{equation}
		 \begin{equation}
			Re(z) = x
		\end{equation}

	\item parte immaginaria:

		\begin{equation}
			z = x + i y\nonumber
		 \end{equation}
		 \begin{equation}
			Im(z) = y
		\end{equation}

    attenzione! la ``parte immaginaria'' di un numero complesso \`e costituita
		da un numero \emph{reale}, il quale viene a sua volta moltiplicato per
		l'operatore immaginario $i$
		

  \item modulo: la ``distanza dal centro'', ossia la somma del quadrato della
	parte reale con il quadrato della parte immaginaria sotto radice:

		 \begin{equation}
		 	z = x + i y\nonumber
		 \end{equation}
		 \begin{equation}
			abs(z) = \sqrt{x^2 + y^2}
		 \end{equation}

		 nel caso di fenomeni periodici, \emph{modulo} e \emph{magnitudine} sono
		 sinonimi

	\item argomento: l'angolo costituito dal numero complesso rispetto all'asse
	reale

		\begin{equation}
		 	z = x + i y\nonumber
		 \end{equation}
		 \begin{equation}
			arg ( z ) = \angle{z} = atan \left ( \frac{Im(z)}{Re(z)} \right ) = atan \left ( \frac{y}{x} \right )
		\end{equation}
	
		 nel caso di fenomeni periodici, \emph{argomento} e \emph{fase} sono
		 sinonimi

\end{itemize}

\paragraph{Esercizi}

\begin{itemize}

  \item rifare la dft con $e$ invece che con $sin$/$cos$
  \item verificare la giustezza della fase 

\end{itemize}

\section{Fasori complessi (cf.\cite[2.4 p.40]{steiglitz1974})}

\begin{itemize}

  \item Abbiamo visto che un oscillatore cosinusoidale pu\`o essere rappresentato come
segue:

\begin{equation}
  F(k) = A cos(\omega k + \phi)\quad \textrm{dove}\,k\,\textrm{\`e un intero}
\end{equation}

(fare il plot di $F(k)$ per valori non--negativi di $k$)

  \item secondo la formula di Eulero,
	
		 \begin{equation}
				F(k) = Re \left ( A e^{i(\omega k + \phi)} \right )
		 \end{equation}

	\item mentre un oscillatore sinusoidale $F(k) = A sin ( \omega k + \phi)$
					corrisponde a
	
		 \begin{equation}
				F(k) = Im \left ( A e^{i(\omega k + \phi)} \right )
		 \end{equation}

	\item quindi $A e^{i(\omega k+ \phi)}$ \`e un \emph{fasore complesso}

  \item interpretazione grafica:
	
		\begin{itemize}

			\item un punto che si muove su un cerchio sul piano
  complesso: A \`e il raggio del cerchio. k sono i punti crescenti sul cerchio

			\item parte reale: coseno campionato

			\item parte immaginaria: seno campionato

			\item $\phi$ \`e il punto di partenza (la fase)

		\end{itemize}

	\item se $\phi = 0$ e $\omega = 0$, il fasore rimane fermo al punto di partenza e $F(k) = A$ (costante reale)

  \item alla fine del giro il fasore ricomincia perch\'e:

		\begin{equation}
  		e^{i(\omega k+ \phi)} = e^{i(\omega k + \phi + 2 \pi)}
		\end{equation}

\end{itemize}

\section{Rivisitando \emph{nyquist} e \emph{foldover} con i fasori complessi}

\begin{itemize}

	\item cosa succede se $\omega = \pi$?

  \item e se $\omega > \pi$?
	
	\item mettiamo conto che $\omega = \pi + x$:

		 \begin{equation}
        e^{i\omega k} = e^{i(\pi + x)k}
		 \end{equation}

  dato che $e^{i(2 \pi k)} = 1$ possiamo aggiungere o togliere $2\pi$ al nostro
  fasore a piacere:

		 \begin{equation}
       e^{i\omega k} = e^{i(\pi + x)k} = e^{i(-2\pi + \pi + x)k} = e^{i(-\pi + x)k}
		 \end{equation}

  quindi sembra che il fasore stia andando all'indietro (effetto stroboscopico)

\end{itemize}

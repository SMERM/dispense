
% $Id: slides.tex 8 2014-02-04 21:01:21Z nicb $
%
% Copyright (C) 2007-2009 Nicola Bernardini nicb@sme-ccppd.org
% 
% This work is licensed under a Creative Commons License, and specifically the
% 
%   Creative Commons Attribution-ShareAlike 2.5 License
%   http://creativecommons.org/licenses/by-sa/2.5/legalcode
% 
% Check http://www.creativecommons.org/ for more information on
% Creative Commons Licenses and the Creative Commons Project.
%
% Set the macros below to whatever is appropriate in a given context
%

\newcommand{\AnnoAccademico}{2014-2015}
\newcommand{\argroot}{../argomenti/informatica}
\newcommand{\beamerslides}{\argroot/beamer}
\newcommand{\imagedir}{\argroot/immagini}
\newcommand{\exampledir}{\argroot/esempi}
\newcommand{\templatedir}{../../../sperimentazione/Padova/1mo-Triennio/Tecnico_di_Sala/corsi/Elettroacustica/latex/beamer/sme-ccppd}
\newcommand{\datadir}{.}
\documentclass[\printmode,compress,xcolor=dvipsnames]{beamer}
\usepackage{ifthen}

\usepackage{beamerthemeSME-CCPPD}
%\usepackage{beamercolorthemeSME-CCPPD}
%\usepackage{beamerinnerthemeSME-CCPPD}

% \ifthenelse{\equal{\printmode}{handout}}%
% {%
% 	\usepackage{pgfpages}
% 	\pgfpagesuselayout{1 on 1}[a4paper,landscape,border shrink=10mm]
% }{}

\includeonlylecture{01}
\newcommand{\lectnum}{(01)}


\usepackage{colortbl}
\usepackage[italian]{babel}
\usepackage{pgf}
\usepackage{xspace}

\usepackage{multimedia}
\usepackage{xmpmulti}
\usepackage{hyperref}
\usepackage[nofancy]{svninfo}
\usepackage{gensymb}
\usepackage{hhline}


\newcommand{\cpyear}{2014, 2015}
\newcommand{\cpholder}{Nicola Bernardini}
\newcommand{\cpholderemail}{n.bernardini@conservatoriosantacecilia.it}

% Use some nice templates

%\beamertemplateshadingbackground{red!10}{structure!10}
\beamertemplatetransparentcovereddynamic
\beamertemplateballitem
\beamertemplatenumberedballsectiontoc

% My colors
\definecolor{notdone}{gray}{0.35}

%\usecolortheme[named=MyColor]{structure}
%\usecolortheme[named=MyColor]{structure}
\beamertemplateshadingbackground{white!10}{white!10}

\input{macros}

\title[INF--0 \lectnum]
{%
  Informatica di Base\lectnum\\[-0.25\baselineskip]
	{\tiny (\rcstag)}
}

\author{%
	Nicola Bernardini\\
    \href{mailto:\cpholderemail}{\cpholderemail}
}
\institute[SMERM]%
{%
	\href{http://www.conservatoriosantacecilia.it}
		 {Conservatorio di Musica ``S.Cecilia'' -- Roma}\\
	\hhref{http://www.conservatoriosantacecilia.it}
}
\date[Roma \AnnoAccademico]{Conservatorio di Roma - A.A. \AnnoAccademico}

\begin{document}
\svnInfo $Id: slides.tex 8 2014-02-04 21:01:21Z nicb $
\newcounter{ms}
\newcounter{lectno}
  
%%%% START %%%%

\begin{frame}
	\titlepage
\end{frame}
  
% ../argomenti/informatica/beamer/funzionamento.tex
% ../argomenti/informatica/beamer/hardware-generico.tex
% ../argomenti/informatica/beamer/hardware-generico-advanced.tex
% ../argomenti/informatica/beamer/rappresentazione-digitale-0.tex
% ../argomenti/informatica/beamer/rappresentazione-digitale.tex
% ../argomenti/informatica/beamer/rappresentazione-digitale-nicb.tex
% ../argomenti/informatica/beamer/sistema-binario.tex
% ../argomenti/informatica/beamer/logica-booleana.tex
% ../argomenti/informatica/beamer/ottale-esadecimale.tex
% ../argomenti/informatica/beamer/tr_vs_td.tex
% ../argomenti/informatica/beamer/funzioni-continue.tex
% ../argomenti/informatica/beamer/ad-conversion.tex
% ../argomenti/informatica/beamer/da-conversion.tex
% ../argomenti/informatica/beamer/campionamento.tex
% ../argomenti/informatica/beamer/campionamento-avanzato.tex
% 
% ../argomenti/informatica/beamer/bufferizzazione.tex
% ../argomenti/informatica/beamer/dithering.tex
% ../argomenti/informatica/beamer/hardware-dedicato-advanced.tex
% ../argomenti/informatica/beamer/hardware-dedicato.tex
% ../argomenti/informatica/beamer/latenza.tex
% ../argomenti/informatica/beamer/quantizzazione.tex

\refstepcounter{lectno}
\lecture{(\arabic{lectno}) Introduzione}{01}
\part{Lezione \arabic{lectno}}
\section{Introduzione}

\slideinput{rappresentazione-digitale-0}
\slideinput{rappresentazione-digitale}
\slideinput{rappresentazione-digitale-nicb}
\slideinput{sistema-binario}
\slideinput{ottale-esadecimale}
\slideinput{logica-booleana}
\slideinput{funzionamento}
\slideinput{hardware-generico}
\slideinput{hardware-generico-advanced}
\slideinput{funzioni-continue}
\slideinput{ad-conversion}
\slideinput{da-conversion}
\slideinput{campionamento}
\slideinput{campionamento-avanzato}
\slideinput{tr_vs_td}

%
% strumentario:
% - emulatore di terminale -> shell
% - command line interface (CLI)
% - interfaccia grafica vs comandi testuali
% - ergonomia, user-friendliness, ecc.
% - barriera della complessità
% - introduzione alla programmazione
% - il mondo del Software Libero
%

\end{document}

%
% Author: Nicola Bernardini <nicb@sme-ccppd.org>
%
% Copyright (c) 2004 Nicola Bernardini
% Copyright (c) 2004 Conservatorio "C.Pollini", Padova
%
% This work is licensed under the Creative Commons 
% Attribution-ShareAlike License. To 
% view a copy of this license, visit 
% http://creativecommons.org/licenses/by-sa/2.0/ 
% or send a letter to Creative Commons, 
% 559 Nathan Abbott Way, Stanford, California 94305, USA.
%
% Some rights reserved.
% CVSId : $Id: rappresentazione-digitale.tex 8 2014-02-04 21:01:21Z nicb $
%
\svnInfo $Id: rappresentazione-digitale.tex 8 2014-02-04 21:01:21Z nicb $

\setcounter{ms}{1}
\begin{frame}
    \frametitle{Rappresentazioni numeriche (\arabic{ms})}

	\begin{itemize}[<+- | alert@+->]

		\item I calcolatori sono un insieme molto elevato di transistors
			bistabili (= che possono emettere solo due valori di tensione)

		\item I calcolatori possono quindi essere assimilati ad
			enormi collezioni di interruttori,
			i quali possono trovarsi in due posizioni sole
			(acceso o spento, oppure 0 o 1)

	\end{itemize}

\end{frame}


\begin{frame}
    \frametitle{Rappresentazioni numeriche (\arabic{ms})}

	\begin{itemize}[<+- | alert@+->]

		\item questi interruttori si chiamano \emph{bit}
			che sta per \emph{BINARY DIGIT}; inoltre:

		\begin{itemize}[<+- | alert@+->]

			\item i \emph{bytes} sono raggruppamenti di 8 \emph{bit}s
			\item i \emph{words} sono raggruppamenti di 16 \emph{bit}s
			\item i \emph{nibbles} sono raggruppamenti di 4 \emph{bit}s

		\end{itemize}

		\item qualsiasi rappresentazione all'interno di
			un elaboratore \`e fatta quindi nei termini
			di sequenze di interruttori che assumono significati
			diversi a seconda del contesto

	\end{itemize}

\end{frame}

\setcounter{ms}{1}
\begin{frame}
    \frametitle{Esempi di Rappresentazione (\arabic{ms})}

	\begin{itemize}[<+- | alert@+->]

		\item una sequenza numerica:\hfill
		\uncover<+->{\tiny
			\begin{tabular}{| c | c |}
				\hline
				\multicolumn{2}{|l|}{\ldots}\\
				\hline
				5 & 0101 \\
				\hline
				7 & 0111 \\
				\hline
				11 & 1011\\
				\hline
				\multicolumn{2}{|l|}{\ldots}\\
				\hline
			\end{tabular}
		}

		\item un carattere:\hfill
		\uncover<+->{\tiny
			\begin{tabular}{| c | c | r |}
				\hline
				'N' & 01001101 & (=78)\\
				\hline
				'i' & 01101001 & (=105)\\
				\hline
				'c' & 01100011 & (=99)\\
				\hline
				\multicolumn{3}{|l|}{\ldots}\\
				\hline
			\end{tabular}
		}

		\item un'istruzione:\hfill
		\uncover<+->{\tiny
			\begin{tabular}{| c | c | p{0.2\textwidth} |}
				\hline
				CMOV & 1000 0000 0000 0000 & (=32768) \\
				     &                     & sposta un numero da una
			                                 locazione di memoria ad
											 un'altra\\
				\hline
			\end{tabular}
		}

	\end{itemize}

\end{frame}

\refstepcounter{ms}
\begin{frame}

    \frametitle{Esempi di Rappresentazione (\arabic{ms})}

	\begin{itemize}[<+- | alert@+->]

		\item Un numero binario pu\`o rappresentare anche altre cose,
			come:

			\begin{itemize}[<+- | alert@+->]
          \item{lo stato di un sistema}

					\only<+->{\begin{center}
              \pgfdeclareimage[width=0.4\textwidth]{status}{\imagedir/status}
              \pgfuseimage{status}
          \end{center}}

				\item una condizione logica
				\only<+->{\scriptsize

					\begin{center}
					\begin{tabular}{| r  c l |}
						\hline
						1 & = & vero\\
						\hline
						0 & = & falso\\
						\hline
					\end{tabular}
					\end{center}
				}

					\only<+->{con la quale \`e possibile
					costruire dei meccanismi del tipo:
          \emph{se il bit $x$ \`e vero (= 1) allora\ldots}}

			\end{itemize}

	\end{itemize}

\end{frame}

\definecolor{im-light-blue}{rgb}{0.8,0.85,1}

\refstepcounter{ms}
\begin{frame}

    \frametitle{Esempi di Rappresentazione (\arabic{ms})}

	\begin{itemize}[<+- | alert@+->]

		\item Un numero binario pu\`o rappresentare anche altre cose,
			come:

			\begin{itemize}[<+- | alert@+->]

        \item{il colore di un pixel (un ``punto'' di uno schermo)}\\
            \uncover<+- | alert@+->{\colorbox{im-light-blue}{\ \ } = \emph{blu
            chiaro} = {\tt 0xccd9ff} = 13477733}

        \item{il valore di una misura di pressione dell'aria}

        \item{la posizione di un elemento in una mappa}

        \item \ldots

			\end{itemize}

	\end{itemize}

\end{frame}

\stepcounter{ms}
\begin{frame}
    
    \frametitle{Esempi di Rappresentazione (\arabic{ms})}

	\begin{itemize}

		\item{Numeri Interi segnati (positivi e negativi):

			\begin{center}
          \pgfdeclareimage[height=0.13\textheight]{sint}{\imagedir/signed-integers}
          \pgfuseimage{sint}
      \end{center}

			e i numeri negativi sono codificati in complemento a
			due, cio\`e invertendo tutti i bit e aggiungendo uno:

			\begin{center}
          \pgfdeclareimage[height=0.13\textheight]{negative}{\imagedir/negative-92}
          \pgfuseimage{negative}
      \end{center}}

		\item{Numeri reali segnati (positivi e negativi):

			\begin{center}
          \pgfdeclareimage[height=0.18\textheight]{float}{\imagedir/floating-point}
          \pgfuseimage{float}
      \end{center}}

	\end{itemize}

\end{frame}

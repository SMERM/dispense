%
% Author: Nicola Bernardini <nicb@sme-ccppd.org>
%
% Copyright (c) 2004 Nicola Bernardini
% Copyright (c) 2004 Conservatorio "C.Pollini", Padova
%
% This work is licensed under the Creative Commons 
% Attribution-ShareAlike License. To 
% view a copy of this license, visit 
% http://creativecommons.org/licenses/by-sa/2.0/ 
% or send a letter to Creative Commons, 
% 559 Nathan Abbott Way, Stanford, California 94305, USA.
%
% Some rights reserved.
% CVSId : $Id: tr_vs_td.tex 8 2014-02-04 21:01:21Z nicb $
%
\svnInfo $Id: tr_vs_td.tex 8 2014-02-04 21:01:21Z nicb $

\setcounter{ms}{0}
\refstepcounter{ms}
\begin{frame}
    \frametitle{Tempo reale e tempo differito (\arabic{ms})}

	\begin{itemize}[<+- | alert@+->]

		\item Al di l\`a delle diverse tipologie,
		      un'ampia distinzione tra i sistemi dedicati alla musica
              si pu\`o tracciare tra:

			  \begin{itemize}[<+- | alert@+->]

				\item sistemi operanti in tempo reale

				\item sistemi operanti in tempo differito

			  \end{itemize}

		\item \emph{Tempo reale}, in questo caso, significa che
			  il periodo necessario al calcolo di uno o pi\`u campioni
			  (un \emph{frame}) \`e \emph{minore} del periodo di campionamento,
			  ossia:

			  \begin{equation}
				\frac{Tempo~di~calcolo}{T_c} < 1 \nonumber
			  \end{equation}

              dove $T_c$ \`e il periodo di campionamento

	\end{itemize}

\end{frame}

\stepcounter{ms}
\begin{frame}
    \frametitle{Tempo reale e tempo differito (\arabic{ms})}

	\begin{itemize}[<+- | alert@+->]

		\item I pro e i contro di tempo reale e tempo differito sono:
		
    \item[~] {\scriptsize
			\begin{tabular}{| p{0.18\textwidth} | p{0.28\textwidth} | p{0.28\textwidth} |}
				\hline
					& {\centering\bfseries Pro} & {\centering\bfseries Contro}\\
				\hline
            Tempo Reale: & rispetta il rapporto causale
							   tra gesto e suono &
							   numero limitato di operazioni\\
							 & feedback immediato &
               dispendioso in termini di risorse di calcolo\\
				\hline
				Tempo Differito & numero illimitato di operazioni &
								  rapporto causale non rispettato
								  (suono senza gesto fisico)\\
							 & realizzabile con risorse di
							   calcolo generiche ed economiche &
                 feedback non immediato\\
        \hline
       \end{tabular}}

	\end{itemize}

\end{frame}

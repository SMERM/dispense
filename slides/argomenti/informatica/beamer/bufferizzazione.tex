%
% Author: Nicola Bernardini <nicb@sme-ccppd.org>
%
% Copyright (c) 2004 Nicola Bernardini
% Copyright (c) 2004 Conservatorio "C.Pollini", Padova
%
% This work is licensed under the Creative Commons 
% Attribution-ShareAlike License. To 
% view a copy of this license, visit 
% http://creativecommons.org/licenses/by-sa/2.0/ 
% or send a letter to Creative Commons, 
% 559 Nathan Abbott Way, Stanford, California 94305, USA.
%
% Some rights reserved.
% CVSId : $Id: bufferizzazione.tex 8 2014-02-04 21:01:21Z nicb $
%
\svnInfo $Id: bufferizzazione.tex 8 2014-02-04 21:01:21Z nicb $

\begin{frame}
	\frametitle{Bufferizzazione}

	\begin{itemize}[<+- | alert@+->]

		\item In sistemi complessi \`e quindi importante
              prevenire le discontinuit\`a di servizio
              del sistema operativo realizzando numerosi
              campioni (ad es. 1024, 2048, ecc.) in anticipo.

		\item Questa tecnica si chiama \emph{bufferizzazione}:

        \begin{figure}[H]
            \begin{center}
				        \pgfimage<+->[height=0.5\textheight]{\imagedir/bufferization}
            \end{center}
            \label{fig:bufferizzazione}
        \end{figure}

	\end{itemize}

\end{frame}

%
% Author: Nicola Bernardini <nicb@sme-ccppd.org>
%
% Copyright (c) 2004 Nicola Bernardini
% Copyright (c) 2004 Conservatorio "C.Pollini", Padova
%
% This work is licensed under the Creative Commons 
% Attribution-ShareAlike License. To 
% view a copy of this license, visit 
% http://creativecommons.org/licenses/by-sa/2.0/ 
% or send a letter to Creative Commons, 
% 559 Nathan Abbott Way, Stanford, California 94305, USA.
%
% Some rights reserved.
% CVSId : $Id: rappresentazione-digitale-0.tex 8 2014-02-04 21:01:21Z nicb $
%
\svnInfo $Id: rappresentazione-digitale-0.tex 8 2014-02-04 21:01:21Z nicb $

\setcounter{ms}{1}
\begin{frame}
  \frametitle{Rappresentabilit\`a (\arabic{ms})}

	\begin{itemize}[<+- | alert@+->]

    \item[~]{~}
    \item{Quante cose si possono rappresentare\dots}

    \item{(\dots o descrivere\dots)}

    \item{\dots con un interruttore?}

    \item{Poche.}

    \item{Essenzialmente: due}

    \item{Due \emph{stati}}

	\end{itemize}

\end{frame}

\refstepcounter{ms}
\begin{frame}
  \frametitle{Rappresentabilit\`a (\arabic{ms})}

	\begin{itemize}[<+- | alert@+->]

		\item Se aggiungiamo un secondo interruttore al primo quanti stati
        possiamo rappresentare?

    \item Quattro:

     \item[~] \begin{tabular}{c | c}
             {\bfseries Int.1} & {\bfseries Int.2}\\
             spento & spento\\
             spento & acceso \\
             acceso & spento \\
             acceso & acceso \\
         \end{tabular}

    \item se ne aggiungiamo un terzo?\ \uncover<+->{8}

    \item e un altro ancora?\ \uncover<+->{16}

    \item{ecc.}

	\end{itemize}

\end{frame}

\refstepcounter{ms}
\begin{frame}

  \frametitle{Rappresentabilit\`a (\arabic{ms})}

	\begin{itemize}[<+- | alert@+->]

    \item quindi ad una progressione aritmetica (l'aggiunta di un interruttore
        alla volta ($+ 1$))\dots

    \item{\dots corrisponde una progressione geometrica ($2^n$) degli stati
        che si possono rappresentare}

    \item{Gi\`a duecento e passa anni fa matematici e filosofi avevano capito che
        questa demoltiplicazione sarebbe tornata molto utile\dots}

    \item{\dots e si sono messi a creare \emph{teorie sulla risolvibilit\`a dei
        problemi}}

	\end{itemize}

\end{frame}

\refstepcounter{ms}
\begin{frame}

  \frametitle{Rappresentabilit\`a (\arabic{ms})}

	\begin{itemize}[<+- | alert@+->]
    
      \item{\emph{Essenzialmente}, queste teorie ruotavano (e ruotano) intorno alla domanda:}

      \item{``Se io ho una quantit\`a sufficientemente grande di interruttori,
          quanti e quali problemi posso risolvere?''}

      \item{Le ultime versioni di queste teorie (nate tra gli anni trenta e la
          seconda guerra mondiale) sono arrivate alla conclusione che\dots}

      \item{\dots con una quantit\`a sufficientemente grande di
          interruttori\dots}

      \item{\dots sono risolvibili tutti i problemi \emph{formalizzabili}}

	\end{itemize}

\end{frame}

\refstepcounter{ms}
\begin{frame}

  \frametitle{Rappresentabilit\`a (\arabic{ms})}

  \vspace{-2ex}
	\begin{itemize}[<+- | alert@+->]
      \item[~]{~}
      \item{solo che\dots}

      \item{\dots certi problemi richiedono una quantit\`a \emph{infinita} di
          interruttori} % turing-complete

      \item{I \emph{computers} nascono da queste teorie.}

      \item{Ahim\'e, alcuni problemi musicali fanno parte -- a quanto
          sembra -- di quest'ultima categoria di problemi}

	\end{itemize}
  
\end{frame}

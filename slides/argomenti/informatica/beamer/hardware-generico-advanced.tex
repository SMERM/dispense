%
% Author: Nicola Bernardini <nicb@sme-ccppd.org>
%
% Copyright (c) 2004 Nicola Bernardini
% Copyright (c) 2004 Conservatorio "C.Pollini", Padova
%
% This work is licensed under the Creative Commons 
% Attribution-ShareAlike License. To 
% view a copy of this license, visit 
% http://creativecommons.org/licenses/by-sa/2.0/ 
% or send a letter to Creative Commons, 
% 559 Nathan Abbott Way, Stanford, California 94305, USA.
%
% Some rights reserved.
% CVSId : $Id: hardware-generico-advanced.tex 8 2014-02-04 21:01:21Z nicb $
%
\svnInfo $Id: hardware-generico-advanced.tex 8 2014-02-04 21:01:21Z nicb $

\stepcounter{ms}
\begin{frame}
    \frametitle{Elementi Hardware (\arabic{ms})}

	\begin{itemize}[<+- | alert@+->]

		\item L'orologio sincronizza tutte le operazioni
			  e le comunicazioni che avvengono sulla
			  scheda madre

		\item le CPU pi\`u recenti usano un multiplo del clock della scheda madre per
			  le istruzioni interne (dato che, in generale,
			  un'istruzione della CPU ha bisogno di pi\`u cicli per essere
			  ultimata)

		\item in linea generale, la CPU accede a tutte le periferiche
		      attraverso delle zone di memoria specializzate
			  dette \emph{registri}, le quali si trovano ad indirizzi ben definiti
			  e separati dalla memoria convenzionale

	\end{itemize}

\end{frame}

\stepcounter{ms}
\begin{frame}
    \frametitle{Elementi Hardware (\arabic{ms})}

	\begin{itemize}[<+- | alert@+->]

		\item il \emph{bus} \`e una collezione di linee elettriche
              in numero sufficiente per ospitare:

			  \begin{itemize}[<+- | alert@+->]

				\item l'indirizzamento di memoria (8, 16, 32 o 64 linee)

				\item i dati in lettura/scrittura dalla memoria

				\item segnali (\emph{interrupts})

				\item ecc.

			  \end{itemize}

	\end{itemize}

\end{frame}

\setcounter{ms}{1}
\begin{frame}
    \frametitle{I Dischi Rigidi (\arabic{ms})}

	\begin{itemize}[<+- | alert@+->]

		\item I dischi rigidi contengono varie parti funzionalmente diverse:

			\vspace{6mm}
			\begin{itemize}[<+- | alert@+->]

				\item la \emph{file allocation table} (\emph{FAT})
                      che definisce

				\item la mappatura tra nomi simbolici dei files
%				      e gruppo di indirizzi ai quali questi files
					  fanno riferimento (\emph{clusters})

				\item la struttura gerarchica dei nomi simbolici
                      (\emph{directories}, \emph{files}, ecc.)

				\item eventuali altre caratteristiche
                      (\emph{links}, ecc.)

			\end{itemize}

	\end{itemize}

\end{frame}

\stepcounter{ms}
\begin{frame}
  \frametitle{I Dischi Rigidi (\arabic{ms})}

	\begin{itemize}[<+- | alert@+->]

		\item I dischi rigidi contengono varie parti funzionalmente diverse:

			\vspace{6mm}
			\begin{itemize}[<+- | alert@+->]

				\item la zona dati, che a sua volta \`e suddivisa in

					\begin{itemize}

						\item settori

						\item blocchi

					\end{itemize}

			\end{itemize}

		\item Un file occupa sempre una dimensione multipla
		      della dimensione dei blocchi.

	\end{itemize}

\end{frame}

\stepcounter{ms}
\begin{frame}{{I Dischi Rigidi (\arabic{ms})}}

	\begin{center}
      \pgfdeclareimage[height=0.75\textheight]{disk}{\imagedir/disk-logical-structure}
      \pgfuseimage{disk}
	\end{center}
	
\end{frame}

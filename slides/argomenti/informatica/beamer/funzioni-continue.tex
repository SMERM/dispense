%
% Author: Nicola Bernardini <nicb@sme-ccppd.org>
%
% Copyright (c) 2004 Nicola Bernardini
% Copyright (c) 2004 Conservatorio "C.Pollini", Padova
%
% This work is licensed under the Creative Commons 
% Attribution-ShareAlike License. To 
% view a copy of this license, visit 
% http://creativecommons.org/licenses/by-sa/2.0/ 
% or send a letter to Creative Commons, 
% 559 Nathan Abbott Way, Stanford, California 94305, USA.
%
% Some rights reserved.
% CVSId : $Id: funzioni-continue.tex 8 2014-02-04 21:01:21Z nicb $
%
\svnInfo $Id: funzioni-continue.tex 8 2014-02-04 21:01:21Z nicb $

\setcounter{ms}{1}
\begin{frame}
    \frametitle{Fenomeni Continui (\arabic{ms})}

	\begin{itemize}[<+- | alert@+->]

		\item Per poter rappresentare e manipolare una variabile
			in continuo cambiamento, un elaboratore ne misura
			il valore ad intervalli costanti (campionamento).

	\end{itemize}

  \vspace{-1ex}
	\pgfimage<+>[width=\textwidth]{\imagedir/wave-sampling-0}
	\pgfimage<+>[width=\textwidth]{\imagedir/wave-sampling-1}
	\pgfimage<+>[width=\textwidth]{\imagedir/wave-sampling-2}

\end{frame}

\refstepcounter{ms}
\begin{frame}
    \frametitle{Fenomeni Continui (\arabic{ms})}

			\begin{center}
				\pgfimage[height=0.85\textheight]{\imagedir/wave-sampling}
			\end{center}

\end{frame}

%
% Author: Nicola Bernardini <nicb@sme-ccppd.org>
%
% Copyright (c) 2004 Nicola Bernardini
% Copyright (c) 2004 Conservatorio "C.Pollini", Padova
%
% This work is licensed under the Creative Commons 
% Attribution-ShareAlike License. To 
% view a copy of this license, visit 
% http://creativecommons.org/licenses/by-sa/2.0/ 
% or send a letter to Creative Commons, 
% 559 Nathan Abbott Way, Stanford, California 94305, USA.
%
% Some rights reserved.
% CVSId : $Id: sistema-binario.tex 8 2014-02-04 21:01:21Z nicb $
%
\svnInfo $Id: sistema-binario.tex 8 2014-02-04 21:01:21Z nicb $

\setcounter{ms}{0}
\refstepcounter{ms}
\begin{frame}
    \frametitle{Il sistema binario (\arabic{ms})}

	\begin{itemize}[<+- | alert@+->]
		\item Funziona esattamente nello stesso modo
			  del sistema decimale, ma con due soli simboli (0-1)
			  al posto di dieci (0-9)
		\item se $3047 = ( 7 \times 10^0 ) +
                      ( 4 \times 10^1 ) +
                      ( 0 \times 10^2 ) +
                      ( 3 \times 10^3 )$
		\item allora $1101 = ( 1 \times 2^0 ) +
                        ( 0 \times 2^1 ) +
                        ( 1 \times 2^2 ) +
                        ( 1 \times 2^3 ) = 13$
    \end{itemize}

\end{frame}

\refstepcounter{ms}
\begin{frame}
    \frametitle{Il sistema binario (\arabic{ms})}

	\begin{itemize}[<+- | alert@+->]

		\item Quindi basta ricordarsi le prime 16 potenze di due
              per poter contare sino a 65535\ldots:
			  \begin{center}
			  {\tiny
			  \begin{tabular}{l p{0.01\textwidth} r |
			                  l p{0.01\textwidth} r |
							  l p{0.01\textwidth} r |
							  l p{0.01\textwidth} r}
				$2^0$ & $=$ & $1$ &
				$2^1$ & $=$ & $2$ &
				$2^2$ & $=$ & $4$ &
				$2^3$ & $=$ & $8$ \\
				$2^4$ & $=$ & $16$ &
				$2^5$ & $=$ & $32$ &
				$2^6$ & $=$ & $64$ &
				$2^7$ & $=$ & $128$ \\
				$2^8$ & $=$ & $256$ &
				$2^9$ & $=$ & $512$ &
				$2^{10}$ & $=$ & $1024$ &
				$2^{11}$ & $=$ & $2048$ \\
				$2^{12}$ & $=$ & $4096$ &
				$2^{13}$ & $=$ & $8192$ &
				$2^{14}$ & $=$ & $16384$ &
				$2^{15}$ & $=$ & $32768$ \\
			  \end{tabular}
			  }
			  \end{center}

		\item il che significa anche che con $n$ bits
              si pu\`o rappresentare $2^{n}$
			  stati (o numeri, o lettere, o istruzioni, ecc.) -- da zero sino a $2^{n-1}$;
		      ad esempio, con 5 bits si pu\`o contare da zero fino
			  a 31, con 14 bits sino 16383,
		      con 27 bits sino a 134.217.727, ecc.

	\end{itemize}

\end{frame}

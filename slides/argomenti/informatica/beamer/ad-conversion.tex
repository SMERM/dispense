%
% $Id: ad-conversion.tex 8 2014-02-04 21:01:21Z nicb $
%
\svnInfo $Id: ad-conversion.tex 8 2014-02-04 21:01:21Z nicb $

\setcounter{ms}{0}
\refstepcounter{ms}
\begin{frame}
    \frametitle{La Conversione Analogico--Digitale (\arabic{ms})}

    \begin{itemize}[<+- | alert@+->]
        \item Molti tipi di convertitori (\arabic{ms}):

            \begin{itemize}[<+- | alert@+->]
                \item convertitori \emph{flash}: banco di comparatori, veloci
                    ma a pochi bit (i comparatori devono essere $2^n - 1$ per $n$ bits)
                \item{convertitori ad approssimazione successiva: un solo
                    comparatore compara iterativamente il segnale in ingresso con quello
                    emesso da un convertitore Digitale--Analogico interno sino
                    a trovare la migliore approssimazione ($==$ la differenza
                    pi\`u piccola)}
            \end{itemize}

    \end{itemize}

\end{frame}

\refstepcounter{ms}
\begin{frame}
    \frametitle{La Conversione Analogico--Digitale (\arabic{ms})}

    \begin{itemize}[<+- | alert@+->]
        \item Molti tipi di convertitori (\arabic{ms}):

            \begin{itemize}[<+- | alert@+->]

                \item{convertitori ad integrazione: un generatore di rampe fa
                    partire una rampa temporizzata e comparata col segnale
                    d'ingresso; quando la rampa raggiunge
                    il valore del segnale d'ingresso
                    viene misurato il tempo}

                \item{convertitori \emph{sigma--delta}: convertitori
                    \emph{flash} sovracampionati che misurano con pochi bit il cambiamento
                    tra valori successivi di un'onda}

            \end{itemize}

    \end{itemize}

\end{frame}

\refstepcounter{ms}
\begin{frame}
    \frametitle{La Conversione Analogico--Digitale (\arabic{ms})}

    \begin{itemize}[<+- | alert@+->]
        \item Problemi di conversione

            \begin{itemize}[<+- | alert@+->]

                \item{\emph{jitter} dell'orologio di campionamento
                    (scostamento dal valore temporale nominale dell'orologio),
                    che introduce non--linearit\`a nelle misurazioni
                    (\emph{aperture error})}

                \item{errore di quantizzazione}

                \item{altre non--linearit\`a legate ad imperfezioni di
                    costruzione}

            \end{itemize}

    \end{itemize}

\end{frame}

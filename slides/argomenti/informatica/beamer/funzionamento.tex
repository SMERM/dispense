%
% Author: Nicola Bernardini <nicb@sme-ccppd.org>
%
% Copyright (c) 2004 Nicola Bernardini
% Copyright (c) 2004 Conservatorio "C.Pollini", Padova
%
% This work is licensed under the Creative Commons 
% Attribution-ShareAlike License. To 
% view a copy of this license, visit 
% http://creativecommons.org/licenses/by-sa/2.0/ 
% or send a letter to Creative Commons, 
% 559 Nathan Abbott Way, Stanford, California 94305, USA.
%
% Some rights reserved.
% CVSId : $Id: funzionamento.tex 8 2014-02-04 21:01:21Z nicb $
%
\svnInfo $Id: funzionamento.tex 8 2014-02-04 21:01:21Z nicb $

\setcounter{ms}{1}
\begin{frame}
    \frametitle{Esempio schematico}

	\begin{itemize}
	
		\item Nello schema che segue viene delineato
			  ci\`o che avviene quando l'utilizzatore
              schiaccia il tasto {\bfseries A} sulla tastiera
              di un computer.

    \item[~]
			  \begin{center}
            \pgfdeclareimage[height=0.65\textheight]{funcexample}{\imagedir/functional-example}
            \pgfuseimage{funcexample}
			  \end{center}

	\end{itemize}

\end{frame}


\begin{frame}
    \frametitle{Sequenza di \emph{boot}}

	\begin{itemize}
	
		\item La ``sequenza di \emph{boot}'' (= calcio d'inizio)
			  di un computer pu\`o essere approssimata come segue:

    \item[~]
			  \begin{center}
            \pgfdeclareimage[height=0.65\textheight]{boot}{\imagedir/boot-sequence}
            \pgfuseimage{boot}
			  \end{center}

	\end{itemize}

\end{frame}

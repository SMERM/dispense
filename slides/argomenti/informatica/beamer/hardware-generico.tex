%
% Author: Nicola Bernardini <nicb@sme-ccppd.org>
%
% Copyright (c) 2004 Nicola Bernardini
% Copyright (c) 2004 Conservatorio "C.Pollini", Padova
%
% This work is licensed under the Creative Commons 
% Attribution-ShareAlike License. To 
% view a copy of this license, visit 
% http://creativecommons.org/licenses/by-sa/2.0/ 
% or send a letter to Creative Commons, 
% 559 Nathan Abbott Way, Stanford, California 94305, USA.
%
% Some rights reserved.
% CVSId : $Id: hardware-generico.tex 8 2014-02-04 21:01:21Z nicb $
%
\svnInfo $Id: hardware-generico.tex 8 2014-02-04 21:01:21Z nicb $

\setcounter{ms}{1}
\begin{frame}
    \frametitle{Elementi Hardware (\arabic{ms})}

	\begin{itemize}[<+- | alert@+->]

		\item Gli elaboratori digitali sono tutti (o quasi)
              costituiti da elementi funzionali simili, quali:

			  \begin{itemize}[<+- | alert@+->]

				\item un orologio (\emph{clock}) che sincronizza tutto
                      il sistema
				\item un circuito ad alta integrazione denominato CPU
				      (\emph{Central Processing Unit})
				\item memoria volatile (\emph{RAM})
				      e permanente (\emph{ROM})
				\item uno o pi\`u dischi rigidi per immagazzinare
				      dati in maniera dinamica

			\end{itemize}

	\end{itemize}

\end{frame}

\stepcounter{ms}
\begin{frame}
    \frametitle{Elementi Hardware (\arabic{ms})}

	\begin{itemize}[<+- | alert@+->]

		\item Altri elementi sono:

			 \begin{itemize}[<+- | alert@+->]
			 \setlength{\itemsep}{4mm}

				\item una o pi\`u interfacce di I/O
				      (\emph{Input/Output} = ingresso/uscita)
					  dedicate a funzioni diverse:

					\vspace{4mm}
					\begin{itemize}[<+- | alert@+->]

						\item video

						\item tastiera

						\item audio

						\item ecc.

					\end{itemize}

				\item un \emph{bus} (=canale) di comunicazione
				      di tutti gli elementi

				\item una scheda madre (MB o \emph{Motherboard})
				      sulla quale risiedono tutti i componenti

			\end{itemize}

	\end{itemize}

\end{frame}

\stepcounter{ms}
\begin{frame}
    \frametitle{Elementi Hardware (\arabic{ms})}

	\begin{center}
      \pgfdeclareimage[height=0.85\textheight]{hardware-schematics}{\imagedir/hardware-schematics}
      \pgfuseimage{hardware-schematics}
	\end{center}

\end{frame}

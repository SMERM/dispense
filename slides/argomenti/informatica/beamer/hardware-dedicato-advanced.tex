%
% Author: Nicola Bernardini <nicb@sme-ccppd.org>
%
% Copyright (c) 2004 Nicola Bernardini
% Copyright (c) 2004 Conservatorio "C.Pollini", Padova
%
% This work is licensed under the Creative Commons 
% Attribution-ShareAlike License. To 
% view a copy of this license, visit 
% http://creativecommons.org/licenses/by-sa/2.0/ 
% or send a letter to Creative Commons, 
% 559 Nathan Abbott Way, Stanford, California 94305, USA.
%
% Some rights reserved.
% CVSId : $Id: hardware-dedicato-advanced.tex 8 2014-02-04 21:01:21Z nicb $
%
\svnInfo $Id: hardware-dedicato-advanced.tex 8 2014-02-04 21:01:21Z nicb $

\setcounter{ms}{1}
\begin{frame}
    \frametitle{Strutture Host/DSP (\arabic{ms})}

		\begin{itemize}[<+- | alert@+->]
		\setlength{\itemsep}{12mm}

			\item Alcuni strumenti in tempo reale sono
                  computers specializzati nell'elaborazione numerica
                  (denominati \emph{Digital Signal Processor}s
                  o \emph{DSP}).

		    \item Tali computers non possiedono sistema
				  operativo proprio (o ne possiedono uno minimale)
				  e sono generalmente controllati da un
				  computer generico (denominato \emph{host}).

		\end{itemize}

\end{frame}

\stepcounter{ms}
\begin{frame}{Strutture Host/DSP (\arabic{ms})}

	\begin{center}
      \pgfdeclareimage[height=0.85\textheight]{hostdsp}{\imagedir/host-dsp-structure}
      \pgfuseimage{hostdsp}
	\end{center}

\end{frame}

\stepcounter{ms}
\begin{frame}
    \frametitle{Strutture Host/DSP}

	\begin{itemize}[<+- | alert@+->]

		\item In una struttura \emph{Host-DSP},
			  il software DSP viene caricato volta per volta
			  dall'\emph{host} nel \emph{DSP},
			  il quale lo esegue ricorsivamente
			  sino al caricamento seguente.

	\end{itemize}

  \only<+->{
	\begin{center}
    \pgfdeclareimage[height=0.6\textheight]{hostdspschema}{\imagedir/host-dsp-schema}
    \pgfuseimage{hostdspschema}
  \end{center}}

\end{frame}

\stepcounter{ms}
\begin{frame}
    \frametitle{Strutture Host/DSP}

	\begin{itemize}[<+- | alert@+->]

		\item Le strutture hardware pi\`u diffuse sono
              attualmente DSP \emph{stand-alone}
              (=autonomi), che assumono generalmente
              la forma di tastiere o expanders.

		\item Strutture del genere sono simili a strutture
              \emph{host-DSP} nelle quali
              il software \`e stato pre-caricato
              e memorizzato in memorie a sola lettura (\emph{ROM}).

	\end{itemize}

	\only<+->{\begin{center}
      \pgfdeclareimage[height=0.45\textheight]{standalone}{\imagedir/stand-alone}
      \pgfuseimage{standalone}
  \end{center}}

\end{frame}

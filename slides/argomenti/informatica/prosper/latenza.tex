%
% Author: Nicola Bernardini <nicb@sme-ccppd.org>
%
% Copyright (c) 2004 Nicola Bernardini
% Copyright (c) 2004 Conservatorio "C.Pollini", Padova
%
% This work is licensed under the Creative Commons 
% Attribution-ShareAlike License. To 
% view a copy of this license, visit 
% http://creativecommons.org/licenses/by-sa/2.0/ 
% or send a letter to Creative Commons, 
% 559 Nathan Abbott Way, Stanford, California 94305, USA.
%
% Some rights reserved.
% CVSId : $Id: latenza.tex 8 2014-02-04 21:01:21Z nicb $
%
\setcounter{ms}{1}
\begin{slide}{Latenza di sistema (\arabic{ms})}
{

	\begin{itemize}

		\item Negli strumenti operanti in tempo reale,
              uno dei parametri pi\`u critici \`e la \emph{latenza}
              del sistema, vale a dire il tempo di risposta
              ai comandi dello strumentista:

	\end{itemize}

	\vspace{5mm}
	\begin{center}
		\includegraphics[height=0.6\textheight]{\imagedir/latency-timing}
	\end{center}

}
\end{slide}

\stepcounter{ms}
\begin{slide}{Latenza di sistema (\arabic{ms})}
{

	\begin{itemize}
	\setlength{\itemsep}{12mm}

		\item Nelle strutture dedicate (tipo \emph{host-DSP}),
              la latenza del sistema \`e generalmente
              fissa e dura pochi cicli macchina
              (cio\`e pochi microsecondi);

		\item Nelle macchine generiche invece,
              la latenza \`e molto pi\`u difficile da prevedere,
              perch\'e i sistemi operativi funzionanti
              su tali macchine sono molto pi\`u complessi

	\end{itemize}
}
\end{slide}

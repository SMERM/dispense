%
% Author: Nicola Bernardini <nicb@sme-ccppd.org>
%
% Copyright (c) 2004 Nicola Bernardini
% Copyright (c) 2004 Conservatorio "C.Pollini", Padova
%
% This work is licensed under the Creative Commons 
% Attribution-ShareAlike License. To 
% view a copy of this license, visit 
% http://creativecommons.org/licenses/by-sa/2.0/ 
% or send a letter to Creative Commons, 
% 559 Nathan Abbott Way, Stanford, California 94305, USA.
%
% Some rights reserved.
% CVSId : $Id: campionamento.tex 8 2014-02-04 21:01:21Z nicb $
%
\setcounter{ms}{1}
\begin{slide}{{\small Frequenza di campionamento (\arabic{ms})}}

	\begin{itemize}

		\item Il teorema di Nyquist impone che la frequenza di
			campionamento sia almeno il doppio della frequenza
			pi\`u alta che si intende rappresentare.

			\begin{center}
				\includegraphics[height=0.24\textheight]{\imagedir/wave-downsamp-1}

				\includegraphics[height=0.24\textheight]{\imagedir/wave-downsamp-2}

				\includegraphics[height=0.24\textheight]{\imagedir/wave-downsamp-3}
			\end{center}

	\end{itemize}

\end{slide}

\stepcounter{ms}
\begin{slide}{{\small Frequenza di campionamento (\arabic{ms})}}
{\scriptsize
	\begin{itemize}

		\item Quando la frequenza di campionamento non \`e sufficiente,
              si verifica il fenomeno del \emph{foldover}
			  (ripiegamento delle frequenze attorno alla met\`a
			  della frequenza di campionamento).

		\item \listento{run: \exampledir/csound_player \exampledir/foldover.csd}{Esperimento:}

			\begin{itemize}

				\item un glissando (prima sinusoidale, poi complesso)
					  va da 4500 Hz a 6500 Hz;
					  su un canale \`e campionato a 44100 Hz,
                      sull'altro a 11025 Hz

					  \begin{center}
						\includegraphics[height=0.5\textheight]{\imagedir/foldover-explained}
					  \end{center}

			\end{itemize}

	\end{itemize}
}
\end{slide}

\stepcounter{ms}
\begin{slide}{{\small Frequenza di campionamento (\arabic{ms})}}
{\scriptsize
	\begin{itemize}

		\item Dato che l'orecchio pu\`o percepire
			  frequenze sino a 20000 Hz ca.,
			  \`e opportuno campionare a frequenze superiori a 40000 Hz.

		\item \listento{run: \exampledir/csound_player \exampledir/downsample.csd }{Esperimento:}

			\begin{itemize}

				\item un brano campionato alle frequenze:
                      1 Hz, 10 Hz, 100 Hz, 980 Hz, 
                      5512.5 Hz, 11025 Hz, 22050 Hz, 44100 Hz

					  \begin{center}
						\includegraphics[height=0.58\textheight]{\imagedir/downsampling-explained}
					  \end{center}
			\end{itemize}

	\end{itemize}
}
\end{slide}

%
% Author: Nicola Bernardini <nicb@sme-ccppd.org>
%
% Copyright (c) 2004 Nicola Bernardini
% Copyright (c) 2004 Conservatorio "C.Pollini", Padova
%
% This work is licensed under the Creative Commons 
% Attribution-ShareAlike License. To 
% view a copy of this license, visit 
% http://creativecommons.org/licenses/by-sa/2.0/ 
% or send a letter to Creative Commons, 
% 559 Nathan Abbott Way, Stanford, California 94305, USA.
%
% Some rights reserved.
% CVSId : $Id: ottale-esadecimale.tex 8 2014-02-04 21:01:21Z nicb $
%
\setcounter{ms}{1}
\begin{slide}{Strumenti di Lettura}

	\begin{itemize}
	\setlength{\itemsep}{6mm}

		\item Sistema ottale (8 simboli, da 0 a 7):
			  permette di raggruppare i bit a 3 a 3,
			  rendendone immediata la lettura/rappresentazione:
				\includegraphics[width=0.36\textwidth]{\imagedir/octal}

		\item Sistema esadecimale (16 simboli, da 0 a F):
			permette di raggruppare i bit a 4 a 4,
			rendendone immediata la lettura/rappresentazione:
				\includegraphics[width=0.48\textwidth]{\imagedir/hexadecimal}

		\item
			Per cui, \`e pi\`u immediatamente evidente che
			${FF5E}_{16} = {1111~1111~0101~1110}_{2}$
			piuttosto che
			$65374_{10} = {1111~1111~0101~1110}_{2}$

	\end{itemize}

\end{slide}

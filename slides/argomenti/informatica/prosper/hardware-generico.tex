%
% Author: Nicola Bernardini <nicb@sme-ccppd.org>
%
% Copyright (c) 2004 Nicola Bernardini
% Copyright (c) 2004 Conservatorio "C.Pollini", Padova
%
% This work is licensed under the Creative Commons 
% Attribution-ShareAlike License. To 
% view a copy of this license, visit 
% http://creativecommons.org/licenses/by-sa/2.0/ 
% or send a letter to Creative Commons, 
% 559 Nathan Abbott Way, Stanford, California 94305, USA.
%
% Some rights reserved.
% CVSId : $Id: hardware-generico.tex 8 2014-02-04 21:01:21Z nicb $
%
\setcounter{ms}{1}
\begin{slide}{{Elementi Hardware (\arabic{ms})}}
{
	\begin{itemize}
	\setlength{\itemsep}{10mm}

		\item Gli elaboratori digitali sono tutti (o quasi)
              costituiti da elementi funzionali simili, quali:

			  \begin{itemize}
			  \setlength{\itemsep}{8mm}

				\item un orologio (\emph{clock}) che sincronizza tutto
                      il sistema
				\item un circuito ad alta integrazione denominato CPU
				      (\emph{Central Processing Unit})
				\item memoria volatile (\emph{RAM})
				      e permanente (\emph{ROM})
				\item uno o pi\`u dischi rigidi per immagazzinare
				      dati in maniera dinamica

			\end{itemize}

	\end{itemize}

}
\end{slide}

\stepcounter{ms}
\begin{slide}{{Elementi Hardware (\arabic{ms})}}
{

	\begin{itemize}

		\item Altri elementi sono:

			 \begin{itemize}
			 \setlength{\itemsep}{4mm}

				\item una o pi\`u interfacce di I/O
				      (\emph{Input/Output} = ingresso/uscita)
					  dedicate a funzioni diverse:

					\vspace{4mm}
					\begin{itemize}

						\item video

						\item tastiera

						\item audio

						\item ecc.

					\end{itemize}

				\item un \emph{bus} (=canale) di comunicazione
				      di tutti gli elementi

				\item una scheda madre (MB o \emph{Motherboard})
				      sulla quale risiedono tutti i componenti

			\end{itemize}

	\end{itemize}

}
\end{slide}

\stepcounter{ms}
\begin{slide}{{Elementi Hardware (\arabic{ms})}}
{

	\begin{center}
		\includegraphics[height=0.95\textheight]{\imagedir/hardware-schematics}
	\end{center}

}
\end{slide}

\stepcounter{ms}
\begin{slide}{{Elementi Hardware (\arabic{ms})}}
{

	\begin{itemize}
	\setlength{\itemsep}{10mm}

		\item L'orologio sincronizza tutte le operazioni
			  e le comunicazioni che avvengono sulla
			  scheda madre

		\item le CPU pi\`u recenti usano un multiplo del clock della scheda madre per
			  le istruzioni interne (dato che, in generale,
			  un'istruzione della CPU ha bisogno di pi\`u cicli per essere
			  ultimata)

		\item in linea generale, la CPU accede a tutte le periferiche
		      attraverso delle zone di memoria specializzate
			  dette \emph{registri}, le quali si trovano ad indirizzi ben definiti
			  e separati dalla memoria convenzionale

	\end{itemize}
}
\end{slide}

\stepcounter{ms}
\begin{slide}{{Elementi Hardware (\arabic{ms})}}
{

	\begin{itemize}

		\item il \emph{bus} \`e una collezione di linee elettriche
              in numero sufficiente per ospitare:

			  \vspace{8mm}
			  \begin{itemize}
			  \setlength{\itemsep}{8mm}

				\item l'indirizzamento di memoria (8, 16, 32 o 64 linee)

				\item i dati in lettura/scrittura dalla memoria

				\item segnali (\emph{interrupts})

				\item ecc.

			  \end{itemize}

	\end{itemize}
}
\end{slide}

\setcounter{ms}{1}
\begin{slide}{{I Dischi Rigidi (\arabic{ms})}}
{

	\begin{itemize}

		\item I dischi rigidi contengono varie parti funzionalmente diverse:

			\vspace{6mm}
			\begin{itemize}
			\setlength{\itemsep}{4mm}

				\item la \emph{file allocation table} (\emph{FAT})
                      che definisce

				\item la mappatura tra nomi simbolici dei files
				      e gruppo di indirizzi ai quali questi files
					  fanno riferimento (\emph{clusters})

				\item la struttura gerarchica dei nomi simbolici
                      (\emph{directories}, \emph{files}, ecc.)

				\item eventuali altre caratteristiche
                      (\emph{links}, ecc.)

			\end{itemize}

	\end{itemize}
}
\end{slide}

\stepcounter{ms}
\begin{slide}{{I Dischi Rigidi (\arabic{ms})}}
{

	\begin{itemize}
	\setlength{\itemsep}{8mm}

		\item I dischi rigidi contengono varie parti funzionalmente diverse:

			\vspace{6mm}
			\begin{itemize}

				\item la zona dati, che a sua volta \`e suddivisa in

					\begin{itemize}

						\item settori

						\item blocchi

					\end{itemize}

			\end{itemize}

		\item Un file occupa sempre una dimensione multipla
		      della dimensione dei blocchi.

	\end{itemize}
}
\end{slide}

\stepcounter{ms}
\begin{slide}{{I Dischi Rigidi (\arabic{ms})}}
{

	\begin{center}
		\includegraphics[height=0.95\textheight]{\imagedir/disk-logical-structure}
	\end{center}
	
}
\end{slide}

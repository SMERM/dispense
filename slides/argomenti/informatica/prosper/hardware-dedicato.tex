%
% Author: Nicola Bernardini <nicb@sme-ccppd.org>
%
% Copyright (c) 2004 Nicola Bernardini
% Copyright (c) 2004 Conservatorio "C.Pollini", Padova
%
% This work is licensed under the Creative Commons 
% Attribution-ShareAlike License. To 
% view a copy of this license, visit 
% http://creativecommons.org/licenses/by-sa/2.0/ 
% or send a letter to Creative Commons, 
% 559 Nathan Abbott Way, Stanford, California 94305, USA.
%
% Some rights reserved.
% CVSId : $Id: hardware-dedicato.tex 8 2014-02-04 21:01:21Z nicb $
%
\setcounter{ms}{1}
\begin{slide}{Hardware dedicato}
{

	\begin{itemize}
	\setlength{\itemsep}{8mm}

		\item Data la potenza di calcolo necessaria
              alle applicazioni musicali,
              molte sono state le soluzioni ``dedicate'' ad esse.

		\item Queste soluzioni si possono raggruppare,
              dalle pi\`u lontane alle pi\`u recenti, in:

			  \begin{itemize}
			  \setlength{\itemsep}{4mm}

			  	\item architetture dedicate host/DSP

				\item sintetizzatori \emph{stand-alone}

				\item stazioni di lavoro musicali

				\item elaboratori personali generici
				      con estensioni multimediali

			  \end{itemize}

	\end{itemize}
}
\end{slide}

\setcounter{ms}{1}
\begin{slide}{Tempo reale e tempo differito (\arabic{ms})}
{

	\begin{itemize}
	\setlength{\itemsep}{2mm}

		\item Al di l\`a delle diverse tipologie,
		      un'ampia distinzione tra i sistemi dedicati alla musica
              si pu\`o tracciare tra:

			  \begin{itemize}

				\item sistemi operanti in tempo reale

				\item sistemi operanti in tempo differito

			  \end{itemize}

		\item \emph{Tempo reale}, in questo caso, significa che
			  il periodo necessario al calcolo di uno o pi\`u campioni
			  (un \emph{frame}) \`e \emph{minore} del periodo di campionamento,
			  ossia:

			  \begin{equation}
				\frac{Tempo~di~calcolo}{T_c} < 1 \nonumber
			  \end{equation}

              dove $T_c$ \`e il periodo di campionamento

	\end{itemize}
}
\end{slide}

\stepcounter{ms}
\begin{slide}{Tempo reale e tempo differito (\arabic{ms})}
{

	\begin{itemize}

		\item I pro e i contro di tempo reale e tempo differito sono:
		
			\vspace{2mm}
			{\scriptsize
			\begin{tabular}{| p{0.2\textwidth} | p{0.3\textwidth} | p{0.3\textwidth} |}
				\hline
					& {\centering\bfseries Pro} & {\centering\bfseries Contro}\\
				\hline
				Tempo Reale: & rispetta il rapporto causale
							   tra gesto e suono &
							   numero limitato di operazioni\\
							 & feedback immediato &
                               dispendioso in termini di risorse di calcolo\\
				\hline
				Tempo Differito & numero illimitato di operazioni &
								  rapporto causale non rispettato
								  (suono senza gesto fisico)\\
							 & realizzabile con risorse di
							   calcolo generiche ed economiche &
                               feedback non immediato\\
				\hline
			\end{tabular}
			}

	\end{itemize}
}
\end{slide}

\setcounter{ms}{1}
\begin{slide}{Strutture Host/DSP (\arabic{ms})}
{
		\begin{itemize}
		\setlength{\itemsep}{12mm}

			\item Alcuni strumenti in tempo reale sono
                  computers specializzati nell'elaborazione numerica
                  (denominati \emph{Digital Signal Processor}s
                  o \emph{DSP}).

		    \item Tali computers non possiedono sistema
				  operativo proprio (o ne possiedono uno minimale)
				  e sono generalmente controllati da un
				  computer generico (denominato \emph{host}).

		\end{itemize}
}
\end{slide}

\stepcounter{ms}
\begin{slide}{Strutture Host/DSP (\arabic{ms})}
{
	\begin{center}
		\includegraphics[height=0.95\textheight]{\imagedir/host-dsp-structure}
	\end{center}
}
\end{slide}

\stepcounter{ms}
\begin{slide}{Strutture Host/DSP}
{

	\begin{itemize}

		\item In una struttura \emph{Host-DSP},
			  il software DSP viene caricato volta per volta
			  dall'\emph{host} nel \emph{DSP},
			  il quale lo esegue ricorsivamente
			  sino al caricamento seguente.

	\end{itemize}

	\begin{center}
		\includegraphics[height=0.6\textheight]{\imagedir/host-dsp-schema}
	\end{center}

}
\end{slide}

\stepcounter{ms}
\begin{slide}{Strutture Host/DSP}
{

	\begin{itemize}

		\item Le strutture hardware pi\`u diffuse sono
              attualmente DSP \emph{stand-alone}
              (=autonomi), che assumono generalmente
              la forma di tastiere o expanders.

		\item Strutture del genere sono simili a strutture
              \emph{host-DSP} nelle quali
              il software \`e stato pre-caricato
              e memorizzato in memorie a sola lettura (\emph{ROM}).

	\end{itemize}

	\begin{center}
		\includegraphics[height=0.45\textheight]{\imagedir/stand-alone}
	\end{center}

}
\end{slide}

\stepcounter{ms}
\begin{slide}{Strutture non-specializzate}
{

	\begin{itemize}

		\item Viceversa, con l'incremento drastico della potenza
		      delle CPU generiche, \`e stato possibile creare
              degli strumenti musicali digitali
              basandosi esclusivamente su elaboratori
              generici (\emph{desktops}, \emph{laptops}, ecc.).

		\item In questi strumenti, tutto i calcoli vengono eseguiti
		      dalla CPU generica e comunicati ad una scheda di I/O audio:

	\end{itemize}

	\begin{center}
		\includegraphics[height=0.4\textheight]{\imagedir/generic-schema}
	\end{center}

}
\end{slide}

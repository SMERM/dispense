%
% $Id: forme_aperte.tex 8 2014-02-04 21:01:21Z nicb $
%
\svnInfo $Id: forme_aperte.tex 8 2014-02-04 21:01:21Z nicb $

\setcounter{ms}{0}
\refstepcounter{ms}
\begin{frame}
    \frametitle{Forme aperte (\arabic{ms})}

    \begin{itemize}

        \item seguendo logiche di indeterminazione completa o parziale della partitura

        \item Esempi:

            \begin{itemize}

                \item la musica di John Cage

                \item Pierre Boulez, \emph{Troisi\`eme Sonate} (...)

                \item Bruno Maderna, \emph{Musica su due dimensioni} (1958)

            \end{itemize}

    \end{itemize}

\end{frame}

\refstepcounter{ms}
\begin{frame}
    \frametitle{Forme aperte (\arabic{ms})}

    \begin{itemize}

        \item improvvisazioni strutturate (gruppi di strumentisti--compositori);

        \item Esempi:

            \begin{itemize}

                \item MEV Musica Elettronica Viva

                \item Gruppo d'improvvisazione Nuova Consonanza

                \item ecc.

            \end{itemize}

    \end{itemize}

\end{frame}

%
% $Id: forme_ternarie.tex 8 2014-02-04 21:01:21Z nicb $
%
\svnInfo $Id: forme_ternarie.tex 8 2014-02-04 21:01:21Z nicb $

\setcounter{ms}{0}
\refstepcounter{ms}
\begin{frame}
    \frametitle{Forme ternarie (\arabic{ms})}

    \begin{itemize}

        \item Con la dissoluzione dell'impianto tonale
              le forme tradizionali perdono il loro impatto espressivo.
              Per questo motivo, queste forme subiscono notevoli scostamenti
              dalle loro accezioni tradizionali.

        \item Forme Ternarie - Esempio:
              Claude Debussy, \emph{Danseuses de Delphes},
              dal primo libro dei preludi (1910).

    \end{itemize}

\end{frame}

\refstepcounter{ms}
\begin{frame}
    \frametitle{Forme ternarie (\arabic{ms})}

    \begin{center}
	    \begin{figure}
	        \pgfimage[height=0.71\textheight]{\imagedir/debussy-1-6}
	        \caption{Claude Debussy, \emph{Danseuses de Delphes}, batt.1-6}
	    \end{figure}
    \end{center}

\end{frame}

\refstepcounter{ms}
\begin{frame}
    \frametitle{Forme ternarie (\arabic{ms})}

    \begin{center}
	    \begin{figure}
        \pgfimage[height=0.71\textheight]{\imagedir/debussy-25-fine}
        \caption{Claude Debussy, \emph{Danseuses de Delphes}, batt.25-31}
      \end{figure}
    \end{center}

\end{frame}

\refstepcounter{ms}
\begin{frame}
    \frametitle{Forme ternarie (\arabic{ms})}

    \begin{center}
        \begin{tabular}{|*{3}{p{0.2\textwidth}|}}
            \hline
                A & B & A'\\
            \hline
                batt.1--10 & batt.11--24 & batt.25--31 (per la verit\`a solo 25--26)\\
            \hline
        \end{tabular}
    \end{center}

    \begin{itemize}
    
        \item nell'esempio, la segmentazione della forma \`e molto sfumata, e basta solo un accenno
            di ripresa per dare il senso ternario

    \end{itemize}

\end{frame}

%
% $Id: alea-principio_estetico.tex 8 2014-02-04 21:01:21Z nicb $
%
\svnInfo $Id: alea-principio_estetico.tex 8 2014-02-04 21:01:21Z nicb $

\setcounter{ms}{0}
\refstepcounter{ms}
\begin{frame}
    \frametitle{Il caso come princio estetico/poetico (\arabic{ms})}

    Grazie al lavoro di John Cage (e in risposta all'iper--determinismo del serialismo integrale)
    negli anni '50 si e` affermata la casualit\`a e l'indeterminazione
    come principo poetico.  Esempi:

    \begin{itemize}

    \item uso dell'\emph{I-Ching}
        (due tiri di tre monete per stabilire un esagramma = 64 combinazioni)
        come principio compositivo in

        \begin{itemize}

            \item \emph{Imaginary Landscape n.4} (1951)

            \item \emph{Williams Mix} (1952)

            \item ecc.

        \end{itemize}

    \end{itemize}

\end{frame}

\refstepcounter{ms}
\begin{frame}
    \frametitle{Il caso come princio estetico/poetico (\arabic{ms})}

    Grazie al lavoro di John Cage (e in risposta all'iper--determinismo del serialismo integrale)

    \begin{itemize}

       \item uso della mappa delle stelle per

        \begin{itemize}

            \item \emph{Atlas Eclipticalis} (1962)

            \item \emph{Etudes Australes} (1974-75)

            \item ecc.

        \end{itemize}

    \end{itemize}

\end{frame}

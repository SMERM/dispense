%
% $Id: limiti_serialismo_integrale.tex 8 2014-02-04 21:01:21Z nicb $
%
\svnInfo $Id: limiti_serialismo_integrale.tex 8 2014-02-04 21:01:21Z nicb $

\setcounter{ms}{0}
\refstepcounter{ms}
\begin{frame}
    \frametitle{Limiti del serialismo integrale}

    \begin{itemize}

        \item I luoghi comuni sulla musica contemporanea
            lasciano intendere che i compositori sarebbero
            stati sopraffatti dalle difficolt\`a
            esecutive e d'ascolto del serialismo integrale.

        \item Piuttosto, l'estremizzazione del serialismo integrale
            e, di converso, la lezione cageana
            hanno mostrato che:

        \begin{itemize}

            \item il massimo della \emph{variabilit\`a}
                (i.e. dei gradi di potenziale variazione,
                o se si vuole il minimo della ridondanza)
                del materiale (offerto con la sistematizzazione
                del serialismo integrale)
                corrisponde all'uniformit\`a percettiva
                (labilit\`a di riferimenti)

            \item la riduzione dell'attivit\`a compositiva
                a schemi altamente combinatori
                conduce paradossalmente al ridimensionamento
                delle aspettative nell'ascolto

        \end{itemize}

    \end{itemize}

\end{frame}

%
% $Id: pentatonic-scales.tex 8 2014-02-04 21:01:21Z nicb $
%
\svnInfo $Id: pentatonic-scales.tex 8 2014-02-04 21:01:21Z nicb $

\setcounter{ms}{0}
\refstepcounter{ms}
\begin{frame}
    \frametitle{Scale pentatoniche (\arabic{ms})}

    Le scale pentatoniche ``classiche''
    possiedono le seguenti caratteristiche:

    \begin{itemize}
        \item cinque note per ottava
        \item costruite con sequenze di seconde maggiori e terze minori
        \item senza semitoni
        \item non permettono una costruzione agevole di accordi tonali
    \end{itemize}

\end{frame}

\refstepcounter{ms}
\begin{frame}
    \frametitle{Scale pentatoniche (\arabic{ms})}

    \begin{center}
        \begin{figure}
            \pgfimage[width=0.9\textwidth]{\imagedir/barbablu-score}
            \caption{Bartok, \emph{Il Castello di Barbabl\`u} (rid. per piano)}
        \end{figure}
    \end{center}

\end{frame}

\refstepcounter{ms}
\begin{frame}
    \frametitle{Scale pentatoniche (\arabic{ms})}

    In tempi pi\`u recenti sono state utilizzate scale pentatoniche
    costruite in altri modi
    (ad es. seconde minori/terze maggiori)

\end{frame}

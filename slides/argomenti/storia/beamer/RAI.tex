%
% $Id: RAI.tex 8 2014-02-04 21:01:21Z nicb $
%
\svnInfo $Id: RAI.tex 8 2014-02-04 21:01:21Z nicb $

\setcounter{ms}{0}
\refstepcounter{ms}
\begin{frame}
    \frametitle{Lo \sfm\ -- Milano (\arabic{ms})}

    \begin{itemize}

        \item Lo \sfm viene creato ufficialmente nel 1955 nella sede della \emph{RAI} di Milano.
            I fondatori sono Bruno Maderna e Luciano Berio,
            allora programmisti RAI in quella sede
            i quali avevano gi\`a creato insieme un lavoro elettronico nel 1954
            (\emph{Ritratto di Citt\`a}), con il quale erano a convincere i dirigenti
            RAI a costituire lo studio.

    \end{itemize}

\end{frame}

\refstepcounter{ms}
\begin{frame}
    \frametitle{Lo \sfm\ -- Milano (\arabic{ms})}

    \begin{itemize}

        \item La caratteristica di questo studio, legata soprattutto alla poliedrica personalit\`a
            del tecnico Marino Zuccheri, era quella di una apertura e di un interesse
            a tutte le variet\`a di elaborazione sonora.

        \item A questa versatilit\`a \`e probabilmente legata
            la sostanziale quantit\`a di composizioni di notevole importanza storica
            create in questo studio, da compositori diversissimi quali
						Pousseur,
						Cage,
						Clementi,
						Castiglioni,
						Donatoni
						ed infine
						Nono.

        \item Lo \sfm \`e stato chiuso nel 1983.

    \end{itemize}

\end{frame}

\refstepcounter{ms}
\begin{frame}
    \frametitle{Lo \sfm\ -- Milano (\arabic{ms})}

    \begin{itemize}

        \item La strumentazione presente nello \sfm rispecchia la diversificazione
            di trattamenti quali:

        \begin{itemize}

            \item oscillatori (all'inizio erano 9)

            \item generatori di rumore

            \item modulatori vari (d'ampiezza, ad anello, ecc)

            \item un pannello di filtri progettati dal fisico Alfredo Lietti

            \item il \emph{tempophon}, un regolatore di tempo e frequenza
                che permetteva di variare la durata del
                tempo di registrazione mantenendo inalterata l'altezza

            \item ecc.

        \end{itemize}

    \end{itemize}

\end{frame}

\refstepcounter{ms}
\begin{frame}
    \frametitle{Lo \sfm\ -- Milano (\arabic{ms})}

    \begin{center}
    \begin{figure}
        \pgfimage[height=0.7\textheight]{\imagedir/RAI-01}
        \caption{Lo \sfm}
    \end{figure}
    \end{center}

\end{frame}

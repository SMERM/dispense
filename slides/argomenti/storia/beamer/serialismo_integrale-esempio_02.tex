%
% $Id: serialismo_integrale-esempio_02.tex 8 2014-02-04 21:01:21Z nicb $
%
\svnInfo $Id: serialismo_integrale-esempio_02.tex 8 2014-02-04 21:01:21Z nicb $

\setcounter{ms}{0}
\refstepcounter{ms}
\begin{frame}
    \frametitle{Serialismo integrale - Esempio 2 (\arabic{ms})}

    \begin{center}
        \begin{figure}
            \pgfimage[height=0.7\textheight]{\imagedir/nono-canto-sospeso-II}
            \caption{Luigi Nono, \emph{Il Canto Sospeso} (1956), II, batt.108--110}
        \end{figure}
    \end{center}

\end{frame}

\refstepcounter{ms}
\begin{frame}
    \frametitle{Serialismo integrale - Esempio 2 (\arabic{ms})}

    Caratteristiche della serie delle note utilizzata:

    \begin{center}
        \pgfimage[height=0.1\textheight]{\imagedir/nono-canto-sospeso-II-row-a}
    \end{center}

    \begin{itemize}

        \item serie ``cuneiforme''

        \item interpretabile anche come:
            \pgfimage[width=0.4\textwidth]{\imagedir/nono-canto-sospeso-II-row-b}
            (un esacordo a cluster seguito dal suo retrogrado
            trasposto ad una quarta aumentata)

        \item in questo movimento ne viene utilizzato solo l'originale

    \end{itemize}

\end{frame}

\refstepcounter{ms}
\begin{frame}
    \frametitle{Serialismo integrale - Esempio 2 (\arabic{ms})}

    \begin{itemize}

        \item Serie dei ritmi:

            \begin{itemize}

                \item serie palindroma basata sui primi sei numeri della
                      dello sviluppo in serie di Fibonacci

                      \begin{center}
                          \begin{tabular}{*{12}{r}}
                            1 & 2 & 3 & 5 & 8 & 13 & 13 & 8 & 5 & 3 & 2 & 1\\
                          \end{tabular}
                      \end{center}

         \end{itemize}

   \end{itemize}

\end{frame}

\refstepcounter{ms}
\begin{frame}
    \frametitle{Serialismo integrale - Esempio 2 (\arabic{ms})}

    \begin{itemize}

        \item Serie dei ritmi (cont.):

        \begin{itemize}

                \item i valori della serie sono distribuiti
                    secondo quattro tipologie di durata:

                    \begin{center}
                        \begin{tabular}{c p{0.15\textwidth} p{0.3\textwidth}}
                            A & croma & Contralto 2\\
                            B & terzina di crome & Soprano 2 - Basso 1\\
                            C & semicrome & Soprano 1 - Soprano 1 + Tenore 2 - Tenore 2\\
                            D & quintine di semicrome & Alto 1 - Basso 2 - Soprano 1\\
                        \end{tabular}
                    \end{center}

                    (segue)

         \end{itemize}

   \end{itemize}

\end{frame}

\refstepcounter{ms}
\begin{frame}
    \frametitle{Serialismo integrale - Esempio 2 (\arabic{ms})}

    \begin{itemize}

        \item Serie dei ritmi (cont.):

              \begin{center}
                  \pgfimage[height=0.5\textheight]{\imagedir/nono-canto-sospeso-II-rhythmic-strand}
                \end{center}

        \item la serie dei ritmi viene permutata circolarmente per
               evitare la corrispondenza ritmo $\Rightarrow$ altezza

   \end{itemize}

\end{frame}

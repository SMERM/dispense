%
% $Id: comporre_col_suono.tex 8 2014-02-04 21:01:21Z nicb $
%
\svnInfo $Id: comporre_col_suono.tex 8 2014-02-04 21:01:21Z nicb $

\setcounter{ms}{0}
\refstepcounter{ms}
\begin{frame}
    \frametitle{Comporre nel/col suono (\arabic{ms})}

    \begin{itemize}

        \item Naturalmente, le tecniche elettroacustiche
            hanno portato numerosi compositori
            a riconsiderare radicalmente
            l'articolazione globale degli elementi musicali,
            e molti sono stati coloro che si sono
            dedicati all'approfondimento 
            delle potenzialit\`a intrinseche dei suoni stessi.

    \end{itemize}

\end{frame}

\refstepcounter{ms}
\begin{frame}
    \frametitle{Comporre nel/col suono (\arabic{ms})}

    \begin{itemize}


        \item Anche in questa visione
            si possono constatare atteggiamenti diversi,
            quali:

        \begin{itemize}

            \item il suono come elemento tematico (esempio sonoro:
                Edgar Var\`ese, \emph{Octandre} (1923), per settimino di fiati
                e contrabbasso)

            \item la composizione come ``lente d'ingrandimento''
                sul suono (esempio sonoro:
                Luigi Nono, \emph{Post-Prae-Ludium per Donau}, (1987),
                per tuba e live-electronics)

            \item il suono come sorgente espressiva (esempio sonoro:
                Jonathan Harvey, \emph{Mortuos plango, vivos voco}, (1980),
                musica elettronica)

            \item il suono come sorgente espressiva (esempio sonoro:
                Giacinto Scelsi, \emph{Pfhat} (1974), per grande orchestra,
                coro e organo)

        \end{itemize}

    \end{itemize}

\end{frame}

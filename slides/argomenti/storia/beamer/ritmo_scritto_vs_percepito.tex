%
% $Id: ritmo_scritto_vs_percepito.tex 8 2014-02-04 21:01:21Z nicb $
%
\svnInfo $Id: ritmo_scritto_vs_percepito.tex 8 2014-02-04 21:01:21Z nicb $

\setcounter{ms}{0}
\refstepcounter{ms}
\begin{frame}
    \frametitle{Ritmo scritto e ritmo percepito (\arabic{ms})}

    Lo iato che separa ritmo scritto da ritmo percepito
    (contrasto/separazione tra metro, battuta e suddivisone)
    viene spesso usato nella musica contemporanea per:

    \begin{itemize}

        \item dissonanza ritmica

        \item raddoppiare il significato ritmico di un dato passaggio

        \item condizionare l'interpretazione

    \end{itemize}

\end{frame}

\refstepcounter{ms}
\begin{frame}
    \frametitle{Ritmo scritto e ritmo percepito (\arabic{ms})}

    \begin{center}
        \begin{figure}
            \pgfimage[width=0.9\textwidth]{\imagedir/webern-op27-2}
            \caption{Anton Webern, \emph{Variationen} Op.27 n.2, batt.1--4}
        \end{figure}
    \end{center}

\end{frame}


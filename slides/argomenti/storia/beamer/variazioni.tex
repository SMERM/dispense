%
% $Id: variazioni.tex 8 2014-02-04 21:01:21Z nicb $
%
\svnInfo $Id: variazioni.tex 8 2014-02-04 21:01:21Z nicb $

\setcounter{ms}{0}
\refstepcounter{ms}
\begin{frame}
    \frametitle{Variazioni come forma (\arabic{ms})}

    \begin{itemize}

        \item Tema e variazioni

        \begin{itemize}

            \item revisione degli schemi, ad esempio:
                Anton Webern, \emph{Variationen op.27} (1936),
                variazioni su una serie dodecafonica
                la cui forma primaria appare nell'ultima variazione,
                e che potrebbero essere riassunte come segue:

                \begin{tabular}{|*{3}{p{0.18\textwidth}|}}
                    \hline
                    Prima & Seconda & Terza\\
                    specularit\`a verticale (figure palindrome) &
                    specularit\`a orizzontale (simmetria acuto-grave) &
                    combinazione delle prime due\\
                    \hline
                \end{tabular}

         \end{itemize}

    \end{itemize}

\end{frame}

\refstepcounter{ms}
\begin{frame}
    \frametitle{Variazioni come forma (\arabic{ms})}

    \begin{itemize}

        \item Variazioni continue (e.g passacaglia, ecc.)

            \begin{itemize}

                \item revisione degli schemi, ad esempio:
                    Olivier Messiaen, \emph{Quatuor pour la fin des temps (Liturgie de Crystal)} (1941)

                    \begin{tabular}{|*{2}{p{0.3\textwidth}|}}
                        \hline
                        talea & color\\
                        \hline
                        17 attacchi & 29 note (10 classi di altezze diverse)\\
                        \hline
                        \multicolumn{2}{|c|}{= ciclo di 493 note}\\
                        \hline
                    \end{tabular}

            \end{itemize}

    \end{itemize}

\end{frame}

%
% $Id: microtonal-scales.tex 8 2014-02-04 21:01:21Z nicb $
%
\svnInfo $Id: microtonal-scales.tex 8 2014-02-04 21:01:21Z nicb $

\setcounter{ms}{0}
\refstepcounter{ms}
\begin{frame}
    \frametitle{Scale microtonali (\arabic{ms})}

    \begin{itemize}

        \item La musica del novecento ha visto una riscoperta
            ed una rivisitazione della microtonalit\`a.
        \item Le scale microtonali hanno le seguenti caratteristiche:

            \begin{itemize}
                \item gli intervalli sono suddivisi in divisioni pi\`u piccole del semitono

                \item possono essere temperate o non-temperate

                \item vengono utilizzate in termini espressivi
                    o integrate in un discorso compositivo pi\`u ampio
            \end{itemize}
    \end{itemize}

\end{frame}

\refstepcounter{ms}
\begin{frame}
    \frametitle{Scale microtonali (\arabic{ms})}

    \begin{center}
        \begin{figure}
            \pgfimage[width=0.9\textwidth]{\imagedir/lutoslawski-livre-score}
            \caption{Witold Lutoslawski, \emph{Livre pour Orchestre} (1968), batt.1-4 (solo Vln I)}
        \end{figure}
    \end{center}
\end{frame}

%
% $Id: superamento_serialita_01.tex 8 2014-02-04 21:01:21Z nicb $
%
\svnInfo $Id: superamento_serialita_01.tex 8 2014-02-04 21:01:21Z nicb $

\setcounter{ms}{0}
\refstepcounter{ms}
\begin{frame}
    \frametitle{\normalsize Tecniche compositive del secondo dopoguerra (3) (\arabic{ms})}

    \begin{itemize}

        \item Durante gli anni '60 gli schemi del serialismo integrale
            elaborati nel decennio precedente hanno incontrato numerosi limiti
            ai quali i compositori hanno trovato soluzioni diverse.

    \end{itemize}

\end{frame}

\refstepcounter{ms}
\begin{frame}
    \frametitle{\normalsize Tecniche compositive del secondo dopoguerra (3) (\arabic{ms})}

    \begin{itemize}

        \item Pi\`u che di superamento in senso stretto
            del serialismo, si tratta di una integrazione
            delle tecniche messe a punto con il
            serialismo all'interno di una strumentazione tecnico-compositiva
            pi\`u vasta.

    \end{itemize}

\end{frame}
    
\refstepcounter{ms}
\begin{frame}
    \frametitle{\normalsize Tecniche compositive del secondo dopoguerra (3) (\arabic{ms})}

    \begin{itemize}

        \item Il segno lasciato dal serialismo
            \`e stato comunque profondo:
            le problematiche che esso cercava di risolvere
            e le soluzioni individuate
            fanno comunque parte del bagaglio odierno
            della composizione contemporanea.

    \end{itemize}

\end{frame}

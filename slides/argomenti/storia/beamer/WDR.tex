
% $Id: WDR.tex 8 2014-02-04 21:01:21Z nicb $
%
\svnInfo $Id: WDR.tex 8 2014-02-04 21:01:21Z nicb $

\setcounter{ms}{0}
\refstepcounter{ms}
\begin{frame}
    \frametitle{Lo \sem\ -- (\arabic{ms})}

    \begin{itemize}

        \item Lo \sem \`e stato costruito nel 1951 negli studi della
            \emph{West-Deutschen Rundfunk} di Colonia per volont\`a
            di Herbert Eimert ed assieme a Robert Beyer e Werner Meyer-Eppler,
            ordinario di Fonetica e Scienza delle Comunicazioni all'Universit\`a di Bonn.

    \end{itemize}

\end{frame}

\refstepcounter{ms}
\begin{frame}
    \frametitle{Lo \sem\ -- (\arabic{ms})}

    \begin{itemize}

        \item L'impronta iniziale data dai suoi fondatori e dai primi compositori
            che utilizzarono lo studio (Stockhausen, Goeyvaerts, K\"onig, ecc.)
            fu estremamente caratterizzata da un atteggiamento rigoroso
            orientato allo sviluppo del serialismo integrale (lo \sem era una sorta
            di controparte elettronica dei \emph{Darmstadter Ferienkurse}).

        \item Conseguentemente, sia l'attenzione compositiva che le tecnologie utilizzate
            erano orientati alla sintesi artificiale dei suoni.

        \item Lo \sem \`e tuttora in funzione.

    \end{itemize}

\end{frame}

\refstepcounter{ms}
\begin{frame}
    \frametitle{Lo \sem\ -- (\arabic{ms})}

    \begin{itemize}

        \item La strumentazione iniziale dello \sem comprendeva:

        \begin{itemize}

            \item uno dei primi registratori a nastro magnetico (il \emph{Magnetophon\ AEG})

            \item il \emph{Melochord} di Harald Bode (una sorta di \emph{Trautonium}
                che possedeva anche un filtro controllabile da una seconda tastiera
                ed un dispositivo per il vibrato)

            \item un generatore di rumore

            \item un oscillatore

            \item un modulatore ad anello

            \item un filtro

            \item ecc.

        \end{itemize}

    \end{itemize}

\end{frame}

\refstepcounter{ms}
\begin{frame}
    \frametitle{Lo \sem\ -- (\arabic{ms})}

    \begin{center}
        \begin{figure}
            \pgfimage[height=0.7\textheight]{\imagedir/WDR-01}
            \caption{Lo \sem negli anni '50}
        \end{figure}
    \end{center}

\end{frame}

\refstepcounter{ms}
\begin{frame}
    \frametitle{Lo \sem\ -- (\arabic{ms})}

    \begin{columns}[T]
        \begin{column}{0.55\textwidth}
			    \begin{itemize}
			
			        \item Durante tutti gli anni '50, nello \sem si costruiscono
			            numerosi strumenti molto peculiari per soddisfare esigenze particolari
			            di alcuni lavori musicali.
			
			        \item Alcuni esempi sono:
			
			        \begin{itemize}
			
			            \item le linee di ritardo variabile realizzate con magnetofoni e nastro magnetico
			
			        \end{itemize}
			
			    \end{itemize}
        \end{column}
        \begin{column}{0.45\textwidth}
            \pgfimage[height=0.8\textheight]{\imagedir/WDR-03}
        \end{column}
    \end{columns}

\end{frame}

\refstepcounter{ms}
\begin{frame}
    \frametitle{Lo \sem\ -- (\arabic{ms})}

    \begin{columns}[T]
        \begin{column}{0.55\textwidth}
			    \begin{itemize}
			
			        \item Alcuni esempi sono (continua):
			
			        \begin{itemize}
			
			            \item una piattaforma girevole di grandi dimensioni
			                per l'altoparlante rotante usato
			                in \emph{Kontakte} di Karlheinz Stockhausen

			        \end{itemize}
          \end{itemize}
       \end{column}
        \begin{column}{0.45\textwidth}
             \pgfimage[height=0.8\textheight]{\imagedir/WDR-02}
       \end{column}
   \end{columns}

\end{frame}

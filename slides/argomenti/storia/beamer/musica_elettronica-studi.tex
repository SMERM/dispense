%
% $Id: musica_elettronica-studi.tex 8 2014-02-04 21:01:21Z nicb $
%
\svnInfo $Id: musica_elettronica-studi.tex 8 2014-02-04 21:01:21Z nicb $

\newcommand{\sem}{\emph{Studio f\"ur Elektronische Musik}\xspace}
\newcommand{\sfm}{\emph{Studi di Fonologia Musicale}\xspace}

\setcounter{ms}{0}
\refstepcounter{ms}
\begin{frame}
    \frametitle{\normalsize Gli Studi di Musica elettronica degli anni '50 (\arabic{ms})}

    \begin{itemize}

        \item Le radio nazionali
            escono rafforzate dopo il secondo conflitto mondiale.
            In molte di esse nascono studi di musica elettronica
            per far fronte a necessit\`a interne quali:

            \begin{itemize}

                \item la realizzazione di laboratori con strumentazione di misura e di manutenzione

                \item la ricerca di nuovi sistemi di produzione del suono

                \item la realizzazione di sigle e stacchi pubblicitari

            \end{itemize}

        \item Dell'utilizzazione e dell'interesse limitati
            da parte delle stesse stazioni radio
            approfittano una serie di giovani ed intraprendenti compositori
            per esplorare le possibilit\`a dei nuovi strumenti elettronici.

    \end{itemize}

\end{frame}

\refstepcounter{ms}
\begin{frame}
    \frametitle{\normalsize Gli Studi di Musica elettronica degli anni '50 (\arabic{ms})}

    \begin{itemize}

        \item Gli studi che sorgono alla fine degli anni '40 e durante tutti gli anni '50
            (entrando in crisi con l'affermazione della televisione
            a discapito della radio) sono molti.

    \end{itemize}

\end{frame}

\refstepcounter{ms}
\begin{frame}
    \frametitle{\normalsize Gli Studi di Musica elettronica degli anni '50 (\arabic{ms})}

    \begin{itemize}
        \item Vale la pena di ricordarne, per la produzione di lavori ormai entrati
            nel repertorio classico, almeno tre:

            \begin{itemize}

            \item Il \emph{Groupe de Recherches Musicales} (GRM)
                sorto ufficialmente all'interno di \emph{Radio France} a Parigi
                nel 1951 (ma gli esperimenti dei suoi fondatori erano cominciati nel 1948)

            \item Lo \sem della \emph{West-Deutschen Rundfunk}
                creato a Colonia nel 1951

            \item Lo \sfm creato nella sede \emph{RAI} di Milano
                nel 1954 (e ufficializzato nel 1955)

            \end{itemize}
        
        \item Queste esperienze sono emblematiche di tre atteggiamenti iniziali
            rispetto alle tecnologie elettroniche.

    \end{itemize}

\end{frame}

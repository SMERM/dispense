%
% $Id: dodecafonia.tex 8 2014-02-04 21:01:21Z nicb $
%
\svnInfo $Id: dodecafonia.tex 8 2014-02-04 21:01:21Z nicb $

\setcounter{ms}{0}
\refstepcounter{ms}
\begin{frame}
    \frametitle{La dodecafonia (\arabic{ms})}

    \begin{itemize}

        \item Funzioni della dodecafonia:

        \begin{itemize}

            \item eliminare metodicamente la gerarchia delle tonalit\`a

            \item ciascuna classe di altezze dell'insieme cromatico
                riveste la stessa importanza

        \end{itemize}

    \end{itemize}

\end{frame}

\refstepcounter{ms}
\begin{frame}
    \frametitle{La dodecafonia (\arabic{ms})}

    \begin{itemize}

        \item Regole classiche del metodo dodecafonico:

        \begin{itemize}

            \item ciascuna classe di altezze dell'insieme cromatico
                pu\`o essere ribattuta, ma non ripetuta prima di aver suonato
                tutte le altre classi

            \item le sequenze dirette di intervalli dell'armonia tonale tradizionale
                (ottave, quinte, ecc.) sono da evitare

       \end{itemize}

    \end{itemize}

\end{frame}

\refstepcounter{ms}
\begin{frame}
    \frametitle{La dodecafonia (\arabic{ms})}

    \begin{itemize}

        \item Impianto costruttivo dodecafonico:

        \begin{itemize}

            \item la serie dodecafonica nelle sue 48 trasformazioni
                (12 trasposizioni per le 4 forme:
                Originale, Inversa, Retrograda, Inversa-Retrograda)

        \end{itemize}

    \end{itemize}

\end{frame}

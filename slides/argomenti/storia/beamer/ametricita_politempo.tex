%
% $Id: ametricita_politempo.tex 8 2014-02-04 21:01:21Z nicb $
%
\svnInfo $Id: ametricita_politempo.tex 8 2014-02-04 21:01:21Z nicb $

\setcounter{ms}{0}
\refstepcounter{ms}
\begin{frame}
    \frametitle{Ametricit\`a e politempo (\arabic{ms})}

    Le modulazioni ritmiche sono un'altra tecnica
    molto utilizzata nelle composizioni del '900.

    Un esempio semplice:

    \begin{center}
        \pgfimage[width=0.99\textwidth]{\imagedir/tempo-modulation}
    \end{center}

\end{frame}

\refstepcounter{ms}
\begin{frame}
    \frametitle{Ametricit\`a e politempo (\arabic{ms})}

    Modulazioni ritmiche molto serrate portano
    all'ametricit\`a caratteristica di molta
    musica contemporanea.
    L'ametricit\`a pone, tra le altre difficolt\`a,
    seri problemi di notazione.

    Esempio risolto con notazione proporzionale:

    \begin{center}
        \begin{figure}
            \pgfimage[width=0.9\textwidth]{\imagedir/berio-sequenza-I}
            \caption{Luciano Berio, \emph{Sequenza I} (1958), primo rigo}
        \end{figure}
    \end{center}

\end{frame}

\refstepcounter{ms}
\begin{frame}
    \frametitle{Ametricit\`a e politempo (\arabic{ms})}

    Esempio di politempo,
    con notazione simultaneamente polimetrica
    e proporzionale:

    \begin{figure}
        \begin{center}
            \pgfimage[width=\textwidth]{\imagedir/daltro-canto}
            \caption{Nicola Bernardini, \emph{D'Altro Canto} (1991), pg.18}
        \end{center}
    \end{figure}

\end{frame}

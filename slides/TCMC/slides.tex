%
% $Id: slides.tex 8 2014-02-04 21:01:21Z nicb $
%
% Copyright (C) 2007-2009 Nicola Bernardini nicb@sme-ccppd.org
% 
% This work is licensed under a Creative Commons License, and specifically the
% 
%   Creative Commons Attribution-ShareAlike 2.5 License
%   http://creativecommons.org/licenses/by-sa/2.5/legalcode
% 
% Check http://www.creativecommons.org/ for more information on
% Creative Commons Licenses and the Creative Commons Project.
%
% Set the macros below to whatever is appropriate in a given context
%

\newcommand{\AnnoAccademico}{2014-2015}
\newcommand{\argroot}{../argomenti/storia}
\newcommand{\beamerslides}{\argroot/beamer}
\newcommand{\imagedir}{\argroot/immagini}
\newcommand{\exampledir}{\argroot/esempi}
\newcommand{\templatedir}{../templates/smerm}
\newcommand{\datadir}{.}
\documentclass[\printmode,compress,xcolor=dvipsnames]{beamer}
\usepackage{ifthen}

\usepackage{beamerthemeSMERM}

% \ifthenelse{\equal{\printmode}{handout}}%
% {%
% 	\usepackage{pgfpages}
% 	\pgfpagesuselayout{1 on 1}[a4paper,landscape,border shrink=10mm]
% }{}

\includeonlylecture{01}
\newcommand{\lectnum}{(01)}


\usepackage{colortbl}
\usepackage[italian]{babel}
\usepackage{pgf}
\usepackage{xspace}

\usepackage{multimedia}
\usepackage{xmpmulti}
\usepackage{hyperref}
\usepackage[nofancy]{svninfo}
\usepackage{gensymb}
\usepackage{hhline}


\newcommand{\cpyear}{2015}
\newcommand{\cpholder}{Nicola Bernardini, Anna Terzaroli, Giuseppe Silvi}
\newcommand{\cpholderemail}{n.bernardini@conservatoriosantacecilia.it}

% Use some nice templates

%\beamertemplateshadingbackground{red!10}{structure!10}
\beamertemplatetransparentcovereddynamic
\beamertemplateballitem
\beamertemplatenumberedballsectiontoc

% My colors
\definecolor{notdone}{gray}{0.35}

%\usecolortheme[named=MyColor]{structure}
%\usecolortheme[named=MyColor]{structure}
\beamertemplateshadingbackground{white!10}{white!10}

\input{macros}

\title[TCMC \lectnum]
{%
  Tecniche Compositive Moderne e Contemporanee \lectnum\\
	{\tiny (\rcstag)}
}

\author{%
	Nicola Bernardini\\con Anna Terzaroli e Giuseppe Silvi\\
	{\tiny \href{mailto:\cpholderemail}{\cpholderemail}}
}
% \institute[SMERM]%
% {%
% 	\href{http://www.conservatoriosantacecilia.it}
% 		 {Conservatorio ``S.Cecilia'' -- Roma}
% }
\date[Roma \AnnoAccademico]{Conservatorio ``S.Cecilia'' Roma - A.A. \AnnoAccademico}

\begin{document}
\svnInfo $Id: slides.tex 8 2014-02-04 21:01:21Z nicb $
\newcounter{ms}
\newcounter{lectno}
  
%%%% START %%%%

\begin{frame}
	\titlepage
\end{frame}
  
\refstepcounter{lectno}
\lecture{(\arabic{lectno}) Tecniche della Musica Contemporanea}{01}
\part{Lezione \arabic{lectno}}
\section{Introduzione}
\subsection[Contesto]{Contesto Storico}
\slideinput{historical_contexts}
\slideinput{time_lines}
\section{Musica}
\subsection[Armonia]{Evoluzione dell'Armonia}
\slideinput{caratteristiche-armonia}
\subsection[Sostituzioni]{Sostituzioni Funzionali}
\slideinput{sostituzioni_funzionali}
\subsection[Cromatismo]{Armonia Cromatica}
\slideinput{armonia_cromatica}
\subsection[Toni Vicini]{Sequenze per Toni Vicini}
\slideinput{toni_vicini}
\subsection[Seq. Reali]{Sequenze Reali}
\slideinput{sequenze_reali}
\subsection[Mov. Paralleli]{Movimenti Paralleli}
\slideinput{movimenti_paralleli}
\subsection[Comp. Accordi]{Composizione degli Accordi}
\slideinput{composizione_accordi}

\refstepcounter{lectno}
\lecture{(\arabic{lectno}) Tecniche della Musica Contemporanea}{02}
\part{Lezione \arabic{lectno}}
\section[Melodia]{Evoluzione della scrittura orizzontale}
\slideinput{caratteristiche-melodia}
\subsection[Modi Diat.]{Modalit\`a Diatonica}
\slideinput{recupero-modale}
\subsection[Scale non-diat.]{Scale non diatoniche}
\slideinput{non-diatonic_scales}
\subsection[Pentatoniche]{Scale Pentatoniche}
\slideinput{pentatonic-scales}
\subsection[Esatonali]{Scale Esatonali}
\slideinput{whole_tone-scales}
\subsection[Ottatoniche]{Scale Ottatoniche}
\slideinput{heptatonic-scales}
\subsection[Chromatiche]{Scale Cromatiche}
\slideinput{chromatic-scales}
\subsection[Microtonali]{Scale Microtonali}
\slideinput{microtonal-scales}
\subsection[Transpos. Lim.]{Modi a trasposizioni limitate}
\slideinput{limited-transpo-modes}
%
\section[Ritmo]{Sviluppi ritmico--metrici}
\slideinput{sviluppi_ritmico-metrici}
\subsection[Tradizione]{Suddivisioni tradizionali}
\slideinput{suddivisioni_tradizionali}
\subsection[Percezione]{Ritmo scritto vs. ritmo percepito}
\slideinput{ritmo_scritto_vs_percepito}
\subsection[Tecniche]{Tecniche contemporanee}
\slideinput{tecniche_metriche_contemporanee}
\slideinput{tecniche_ritmiche_contemporanee}
\subsection[Politempo]{Ametricit\`a e politempo}
\slideinput{ametricita_politempo}

\refstepcounter{lectno}
\lecture{(\arabic{lectno}) Tecniche della Musica Contemporanea}{03}
\part{Lezione \arabic{lectno}}
\section[Forma]{Evoluzione della forma}
\slideinput{caratteristiche-forma}
\subsection[Ternarie]{Forme ternarie}
\slideinput{forme_ternarie}
\subsection[Arco]{Forme rond\`o e forme ad arco}
\slideinput{forme_rondo_arco}
\subsection[Sonata]{Forme sonata}
\slideinput{forma_sonata}
\subsection{Variazioni}
\slideinput{variazioni}
\subsection{Canoni e fughe}
\slideinput{canoni_fughe}
\subsection[Geometria]{Forme derivate da suggestioni geometriche}
\slideinput{suggestioni_geometriche}
\subsection[Aperte]{Forme aperte}
\slideinput{forme_aperte}
\subsection[Non--narrativit\`a]{Approcci non--narrativi}
\slideinput{approcci_non_narrativi}
\section{Atonalit\`a}
\slideinput{atonalita}
\subsection{Esempio}
\slideinput{atonalita_esempio}
\section{Dodecafonia}
\slideinput{dodecafonia}
\subsection{Esempi}
\slideinput{dodecafonia-esempio-01}
\slideinput{dodecafonia-esempio-02}

\refstepcounter{lectno}
\lecture{(\arabic{lectno}) Tecniche della Musica Contemporanea}{04}
\part{Lezione \arabic{lectno}}
\section[Secondo Dopoguerra]{Tecniche compositive del secondo dopoguerra}
\subsection{Serialismo Integrale}
\slideinput{secondo_dopoguerra-serialismo_integrale}
\slideinput{serialismo_integrale-esempio_01}
\slideinput{serialismo_integrale-esempio_02}
\subsection{Alea}
\slideinput{secondo_dopoguerra-alea}
\slideinput{alea-principio_estetico}
\slideinput{alea-esempio_01}
\subsection{Indeterminazione}
\slideinput{indeterminazione}
\slideinput{indeterminazione-esempio_01}
\subsection{Statistica}
\slideinput{processi_probabilistici}
\slideinput{processi_probabilistici-esempio_01}

\refstepcounter{lectno}
\lecture{(\arabic{lectno}) Tecniche della Musica Contemporanea}{05}
\part{Lezione \arabic{lectno}}
\section[Dopo il serialismo]{Superamento del Serialismo Integrale}
\slideinput{superamento_serialita_01}
\slideinput{limiti_serialismo_integrale}
\slideinput{superamento_serialismo_integrale}
\subsection[Nuova Espressivit\`a]{Recupero di Elementi Espressivi}
\slideinput{recupero_elementi_espressivi}
\subsection[Nuovo Tematismo]{Nuove Forme Tematiche}
\slideinput{nuove_forme_tematiche}
\subsection[Elementi Connotativi]{Utilizzazione di Elementi Connotativi}
\slideinput{elementi_connotativi}
\subsection[Il Suono]{Comporre nel/col Suono}
\slideinput{comporre_col_suono}
\subsection[Conclusioni]{Riflessioni Conclusive}
\slideinput{riflessioni_conclusive}

\refstepcounter{lectno}
\lecture{(\arabic{lectno}) Tecniche della Musica Contemporanea}{06}
\part{Lezione \arabic{lectno}}
\section[Musica Elettronica]{Gli Studi di Musica Elettronica}
\slideinput{musica_elettronica-studi}
\subsection{GRM}
\slideinput{GRM}
\begin{frame}
    \frametitle{GRM -- Esempio Sonoro}

    \begin{itemize}

        \item Pierre Scheffer, \emph{Etude aux Chemins de Fer} (1948)

    \end{itemize}

\end{frame}
\subsection{WDR}
\slideinput{WDR}
\begin{frame}
    \frametitle{WDR -- Esempio Sonoro}

    \begin{itemize}

        \item Karlheinz Stockhausen, \emph{Komposition n.2 -- Studie n.1} (1952)

    \end{itemize}

\end{frame}
\subsection{RAI}
\slideinput{RAI}
\begin{frame}
    \frametitle{RAI -- Esempio Sonoro}

    \begin{itemize}

        \item RAI: Luciano Berio, \emph{THEMA -- Omaggio a Joyce} (1958)

    \end{itemize}

\end{frame}

\end{document}
